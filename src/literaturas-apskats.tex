\chapter{Literatūras apskats}

Ir aprakstītas vairākas metodes, lai veiktu cilvēku plūsmu analīzi attēlos un video fragmentos \cite{brostow2006unsupervised,chen2013cumulative,ge2009marked,chen2015person,lempitsky2010learning}. Vispārīgi, pirmie pētījumi tika vērsti uz detektēšanas veida izvēli un vai problēmu var risināt izmantojot segmentācijas metodes \cite{tu2008unified}. Šīs metodes nelabvēlīgi ietekmēja objektu nostāšanās vienam aiz otra, objektu pazušana un nekārtīgs (pārblīvēts TODO pārbaudīt tulkojumu (high clutter background)) fons attēlos. Jaunākos risinājumus var vispārīgi sadalīt trīs kategorijās: risinājumi, kas balstīti uz regresiju, risinājumi, kas balstīti uz pūļa blīvuma novērtējumu un risinājumi, kas balstīti uz konvolūcijas neironu tīkliem. 

\subsubsection{Risinājumi, kas balstīti uz regresiju}
Lai novērstu objektu paslēpšanos attēlā un pārblīvēta fona problēmas, pētnieki mēģināja skaitīt cilvēkus izmantojot regresiju. Parasti, regresija šādos risinājumos tiek veikta starp dažādām attēla īpašībām un objektu skaitu. Šāds risinājums sadala attēlu vairākos mazākos attēlos un katram šim mazajam attēlam tiek veikta aptuvenā skaita noteikšana izmantojot segmentācijas metodes. Lai noteiktu kopējo cilvēku skaitu attēlā, ir jāsaskaita katra mazā attēla aptuvenie novērtējumi. Lai apmācītu šādu sistēmu, pirmais no attēla apstrādes posmiem ir atmest attēla fonu un veikt \textit{ground-truth} novērtējumu, kas nozīmē, ka tiek manuāli izskaitīts cilvēku skaits katrā sadalītajā attēlā  \cite{chan2009bayesian,ryan2009crowd,chen2012feature}.

Šāda pieeja tika izveidota, pieņemot, ka dotajai cilvēku skaitīšanas sistēmai būtu vieglāk novērtēt cilvēku daudzumu katrā grupā atsevišķi, nevis novērtēt cilvēku daudzumu visam pūlim vienlaikus. Pieņemot, ka attēlā ir pūlis ar 20 cilvēkiem, šo pūli var sadalīt divās lielās grupās vai arī desmit pāros. Ņemot šādu attēlu vispārīgi, šāds pūļa sadalījums var nebūt tik skaidri novērtējams, jo attēlam vispārīgi būtu daudz vairāk atšķirīgu īpašību. Eksistējošas metodes, kas izmanto visu attēlu no attēla iegūst daudz vairāk īpašību, kas nozīmē, ka ir nepieciešami vairāk apmācības datu \cite{chan2008privacy}. Pētot citus rakstus var secināt, ka regresijas metodes parasti sastāv no divām lielām komponentēm: zema līmeņa īpašību iegūšanas un regresijas modeļa implementēšanas \cite{xiong2017spatiotemporal}. 

\subsubsection{Risinājumi, kas balstīti uz pūļa blīvuma novērtējumu}
Lai gan pūļu skaitīšanas risinājumi, kas balstīti uz regresiju veiksmīgi tika galā ar objektu pārklāšanās un pārblīvētā fona problēmām, tika palaista garām svarīga telpiska informācija, jo regresija tika izmantota uz lokālajām īpašībām (katra sadalītā attēla īpašībām atsevišķi). Pētījumā \cite{lempitsky2010learning} tiek piedāvāts jauns veids kā iemācīties lineāras attiecības starp sadalīto attēlu īpašībām un attiecīgās objektu blīvuma kartes izmantojot regresiju. Novērojot, ka iemācīties lineāras attiecības ir sarežģīts uzdevums, tika izveidots risinājums, kas piedāvā iemācīties nelineāras attiecības starp sadalīto attēlu īpašībām un objektu blīvuma kartēm izmantojot nejaušo mežu ietvaru (no angļu val. \textit{random forest framework}) \cite{pham2015count}. Vairāki mūsdienu risinājumi piedāvā metodes, kas balstās uz blīvuma karšu regresiju \cite{wang2016fast,xia2016block}. 

\subsubsection{Risinājumi, kas balstīti uz konvolūcijas neironu tīkliem}
Tā kā mūsdienās klasifikācijā un atpazīšanas uzdevumu risināšanā ļoti veiksmīgi darbojas uz konvolūcijas neironu tīkliem balstītas metodes. Pētnieki ir izveidojuši CNN, ar mērķi veikt pūļa skaitīšanu un blīvuma novērtējumu \cite{wang2015deep,shang2016end,walach2016learning}. Pretēji jau eksistējošajām metodēm, kas pūļa skaitu novērtē izmantojot attēla sadalīšanas metodes, Šangs \cite{shang2016end} piedāvā metodi, kas veic novērtējumu izmantojot CNN. Šī metode novērtējumu veic paralēli skaitot cilvēkus gan globālajā kontekstā, gan lokālajā kontekstā. Žangs \cite{zhang2016single} piedāvāja daudz-kolonnu arhitektūru, kas izgūst īpašības dažādos mērogos. Līdzīgi šai metodei, tika izveidots skaitīšanas modelis, kas veica novērtējumu pūļa blīvuma kartēm, ko nosauca par \textit{Hydra CNN} \cite{onoro2016towards}. Pētnieks Mardsens \cite{marsden2017resnetcrowd} pētīja pilnīgos konvolūciju neironu tīklus un vairākuzdevumu (no ang. val. \textit{multi-task}) apmācību, kuru apvienojot veica cilvēku skaitīšanu. Minētie vairākuzdevumu apmācības un daudz-kolonnu risinājumi ir sasnieguši labus rezultātus, uzrādot salīdzinoši zemu skaitīšanas kļūdu. Balstoties uz minētajiem risinājumiem var izdarīt sekojošus novērojumus \cite{sindagi2017generating}:
\begin{itemize}
	\item Šīs metodes neietver kontekstuālu informāciju, kas ir svarīgi, lai iegūtu labākus rezultātus;
	\item Lai gan eksistējošie risinājumi izmanto regresiju ar pūļu blīvuma kartēm, šie risinājumi ir vairāk balstīti uz skaitīšanas kļūdas samazināšanu, nevis blīvuma karšu kvalitātes uzlabošanu;
\end{itemize}

Šī darba ietvaros tiks veikta skaitīšana pēc detektēšanas. Skaitīšana pēc detektēšanas nozīmē, ka tiks izmantots atsevišķs objektu detektēšanas risinājums, kas lokalizēs pēc konvolūciju neironu tīkla klasifikācijas atrastās klases. Tad, kad lokalizētas visas objekta instances, skaitīšanas uzdevums paliek elementārs. Taču objektu detektēšanas problēmām nav pilnībā atrasti risinājumi, it īpaši, ja vairākas objektu instances attēlā pārklājas.// šite jāturpnia no Learning To Count Objects in Image