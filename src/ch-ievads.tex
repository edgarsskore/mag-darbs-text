\chapter{Ievads}
Jebkāda veida objektu atpazīšana, neatkarīgi no tā vai tās ir galvas, cilvēki vai jebkāds cits objekts, ir svarīgs uzdevums dažādām datorredzes problēmām. Sejas atpazīšanas problēmas ir zināmas jau diezgan ilgu laiku, taču cilvēku atpazīšana dažādos kameru leņķos, pie dažādām izšķirtspējām, pie dažādiem apgaismojumiem, dažādām cilvēku pozām vai cilvēku atpazīšana attēlos, kuri ir pārpildīti ar cilvēkiem joprojām ir sarežģīts uzdevums.

Šī darba nolūkos, autors izvirzīja mērķi izstrādāt risinājumu, kas veic cilvēku plūsmas analīzi, izmantojot konvolūcijas tīklu un sekošanas algoritmu analīzi. Lai sasniegtu darba mērķi, ir nepieciešams veikt sekojošus uzdevumus:
\begin{itemize}
	\item Iepazīties ar plūsmas analīzes metodoloģiju, veikt literatūras izpēti;	
	\item Salīdzināt dažādas konvolūcijas neironu tīklu arhitektūras un objektu detektēšanas algoritmus;
	\item Salīdzināt dažādus sekošanas algoritmus;
	\item Izstrādāt risinājumu, kas veic objektu detektēšanu un sekošanu šiem objektiem video kadros;
	\item Pārbaudīt izveidotā risinājuma veiktspēju ar dažādiem video fragmentiem.	
\end{itemize}

Autors pētījumā plāno izpētīt metodes, lai varētu novērot cilvēku plūsmu pilsētu objektos, kuri ir nesen izveidoti vai ir nesen tikuši atjaunoti vai remontēti, lai noteiktu ceļu, velosipēdu brauktuvju vai pastaigu taku nolietojumu. Rezultātā tiks iegūta sistēma, kas uzticami varētu veikt cilvēku skaitīšanu video ierakstos vai arī tiešsaistes straumē izmantojot konvolūciju neironu tīklus. Tālāk šādu sistēmu būtu iespējams viegli modificēt ,lai to pielietotu objektu skaitīšanai citās nozarēs, piemēram, bioloģijā (šūnu skaitīšana) vai automašīnu skaitīšana.

Esošās metodes, kas balstās uz konvolūciju neironu tīkliem veiksmīgi tiek galā ar dažām no augstākminētajām problēmām, taču attēli, kuros ir ļoti daudz cilvēku joprojām sagādā grūtības un, lai atrastu visus cilvēkus tādos attēlos ir nepieciešamas pielietot papildu metodes. Divas no populārākajām metodēm ir tiešā un netiešā cilvēku skaitīšana. Netiešās cilvēku skaitīšanas galvenā ideja ir izmantot kontekstuālo (kopējo) informāciju. Vairāki eksistējošie pētījumi, kuros izmanto semantisko segmentāciju, ainu pārsēšanu un vizuālās īpašības demonstrē, ka kontekstuālās informācijas izmantošana nodrošina nozīmīgus uzlabojumus rezultātos, kas nozīmē, ka kontekstuālās informācijas izmantošana var palīdzēt apmācības procesā un rezultātā ļaus iegūt prezīcāku cilvēku skaita novērojumu \cite{sindagi2017generating}.
Otrā no populārākajām metodēm, tiešā cilvēku skaitīšana, katrā ainā meklē visas cilvēku sejas pa vienai balstoties uz sejas īpašībām (no angļu val. - \textit{feature based}) un tās saskaita, kopējo attēla kontekstu neņemot vērā. 

TODO aprakstīt darba struktūru 

\section{Mašīnmācīšanās pamatjēdziens}
Mašīnmācīšanās algoritmi ir kļuvuši par neatņemamu sastāvdaļu programmatūras izstrādātājiem un kompānijām, kas savas aplikācijas grib padarīt "gudras". Lai arī cik, mūsdienās, šis jēdziens ir kļuvis populārs, oficiāla mašīnmācīšanās definīcija nav noteikta. Visvienkāršākais mašīnmācīšanās pielietojums ir apstrādāt datus, mācīties no tiem un no iegūtajiem rezultātiem pieņemt lēmumus vai veikt minējumus reālās pasaules problēmu risināšanai. Tā vietā, lai manuāli veidotu programmatūras risinājumus, kas veic kādu uzdevumu, tiek apmācīti datori vai citas ierīces, izmantojot lielus datu apjomus. 
Pasaulē ir daudz un dažādi mašīnmācīšanās algoritmi un katru dienu tiek publicēti simtiem jaunu algoritmu. Tos var sagrupēt pēc apmācības veida (vadītā apmācība (\textit{supervised learning}), nevadītā apmācība (\textit{unsupervised learning}), pusvadītā apmācība (\textit{semi-supervised learning})) kā arī pēc formas vai funkcijas līdzībām (klasifikācija, regresija, lēmumu koki (\textit{decision trees}), klasterēšana (\textit{clustering}), dziļā mašīnmācīšanās (\textit{deep learning})). Neatkarīgi no apmācības veida vai pielietojuma, visas mašīnmācīšanās algoritmu kombinācijas sastāv no klasifikatoriem (atbalsta vektora mašīna, lēmuma koki, neironu tīkli), vērtēšanas funkcijām (varbūtības funkcijas, robežfunkcijas, izmaksu funkcija) un optimizācijas funkcijām (mantkārīgā meklēšana, nepārtrauktās optimizācijas metodes (\textit{continuous optimization})). Izmantojot šīs sastāvdaļas, mašīnmācīšanās algoritmu pamata mērķis ir būt spējīgam funkcionēt ne tikai ar apmācībā piedāvātajiem datiem, bet arī spēt darboties ar datiem, ar kuriem algoritms nav saskāries. Atkarībā no veicamā uzdevuma, ir dažādi veidi kā panākt, lai datori vai jebkura cita ierīce mācās, sākot ar visparastākajiem lēmumu kokiem, beidzot ar ģenētiskajiem algoritmiem un mākslīgajiem neironu tīkliem. 


\newpage
\section{Datorredze}
Datorredze (no angļu val. \textit{computer vision}) ir datorzinātņu nozare, kuras mērķis ir ļaut datoriem redzēt un veikt tādus pašus uzdevumus kā cilvēki veiktu ar acīm un darīt to tikpat efektīvi. Redzes nodrošināšana datoriem nozīmē dot tiem spēju identificēt un apstrādāt attēlus līdzīgi kā to spēj darīt cilvēki. Tas ir kā nodot cilvēku inteliģenci un instinktus datoram. Datorredzes sistēmas parasti iedala trīs komponentēs:
\begin{itemize}
	\item Attēla iegūšana;
	\item Attēla apstrāde;
	\item Attēla analīze;
\end{itemize}
Līdzīgi kā cilvēku pasaules izpratne balstās uz spēju pieņemt lēmumus ņemot vērā redzēto, piedāvājot datoriem šādu vizuālu izpratni, tiem būtu iespējams pieņemt patstāvīgus lēmumus.

\begin{figure}[h]%
	\centering
	\includegraphics[height=2cm]{images/computervision1.png} %
	\caption{Datorredzes pamatprincips}%
	\label{fig:example}%
\end{figure}

Attēlu iegūšana ir process kurā reālās pasaules notikumi tiek pārveidoti bināros datos, kurus interpretē kā digitālus attēlus vai kā daļu no video fragmenta. 

Attēlu apstrāde ir iegūto attēlu zema līmeņa apstrāde. Pirmajā solī iegūtajiem binārajiem datiem pielieto algoritmus, kas norāda uz attēla daļām, kas satur zema līmeņa informāciju. Šādu informāciju var izšķirt pēc jebkādiem ģeometriskiem elementiem, kas sastāda attēlu, piemēram, punkti attēlā, attēla malas vai segmenti. Zema līmeņa attēlu apstrādes algoritmi ir malu detektēšana, segmentācijas algoritmi, klasifikācija, īpašību detektēšana.
\begin{figure}[h]%
	\centering
	\includegraphics[height=3cm]{images/computervision2.png} %
	\caption{Ābolu segmentācija attēlā \cite{compv1}}%
	\label{fig:example}%
\end{figure} 

Pēdējā datorredzes sistēmu komponente ir attēlu analīzes solis, kurā notiks attēla analīze un pēc šī soļa datorredzes sistēmai būs iespējams pieņemt lēmumu un izvadē to atgriezt. Attēlu analīzes solī tiek pielietoti augsta līmeņa algoritmi, ņemot vērā gan attēla apstrādes solī iegūto zema līmeņa informāciju, gan pašu attēlu. Piemēri kur var izmantot šādu augsta līmeņa attēlu analīzi ir trīsdimensiju ainu atveidošana, objektu atpazīšana, objektu sekošana, cilvēku plūsmas analīze.
\begin{figure}[h]%
	\centering
	\includegraphics[height=4cm]{images/computervision3.png} %
	\caption{Objektu detektēšana pēc segmentācijas pielietošanas \cite{compv2}}%
	\label{fig:example}%
\end{figure} 

Izstrādājot datorredzes sistēmas, pētnieki saskaras ar dažādām problēmām un izaicinājumiem. Parasti šīs problēmas ir atkarīgas no datu kvalitātes, sistēmas pielietojuma un apkārtējās pasaules ietekmes uz datiem un aparatūru. Datorredzes pētnieki izstrādā risinājumus, lai padarītu datorredzes algoritmus stabilākus un efektīvākus sarežģītos uzstādījumos: nekvalitatīvi vai trokšņaini dati, reālā laika apstrāde un ierobežota skaitļošanas jauda. Mūsdienās, lai risinātu šīs problēmas, tiek savienoti mašīnmācīšanās risinājumi ar datorredzes risinājumiem.

Klasiskie datorredzes algoritmi ir smalki pētīti un optimizēti, lai iegūtu labāko veiktspēju un lai tie efektīvi izmantotu datora skaitļošanas resursus, kamēr mašīnmācīšanās algoritmi piedāvā precīzākus un vispusīgākus risinājumus, taču prasa lielus skaitļošanas resursus. Ņemot vērā iepriekš minēto, mūsdienu pētījumos ir populāri risinājumi, kas apvieno standarta datorredzes algoritmus un mašīnmācīšanās risinājumus. Labs piemērs abu šo nozaru apvienošanā ir kustīgu objektu meklēšana video fragmentos. Lai iegūtu augstāku precizitāti un taupītu skaitļošanas resursus ir iespējams attēla apstrādi veikt ar datorredzes algoritmiem un attēla analīzi (klasifikāciju, lokalizāciju, sekošanu) veikt ar neironu tīkliem.



