%% Nenumurēta nodaļa, kas uzrādās satura rādītājā
\chapter*{Secinājumi un priekšlikumi}
\addcontentsline{toc}{chapter}{Secinājumi un priekšlikumi}
\begin{enumerate}
\item \textbf{Secinājumi un priekšlikumi} jāraksta tēžu veidā.
\item Secinājumiem jāatspoguļo svarīgākās atziņas, kas izriet no pētījuma, satur atbildes uz ievadā izvirzīto mērķi un uzdevumiem.
\item Secinājumos jāpaskaidro veiktā pētījuma tautsaimnieciskā, zinātniskā vai praktiskā nozīme un autora personīgais veikums uzdevuma risināšanā.
\item Secinājumus nedrīkst pamatot ar datiem un faktiem, kas nav minēti darbā.
\item Secinājumos nav pieļaujami citāti no citu autoru darbiem, tajos jāatspoguļo tikai darba autora domas, spriedumi, atziņas.
\item Priekšlikumiem jāizriet no darbā veiktajiem pētījumiem un izdarītajiem secinājumiem, tiem jābūt konkrētiem un pamatotiem.
\item Priekšlikumos apkopo arī darbā pamatotās rekomendācijas trūkumu novēršanai.
\item Secinājumi un priekšlikumi jānumurē ar arābu cipariem.
\end{enumerate}
