\documentclass[12pt,paper=a4]{report}
\usepackage{fontspec}
\usepackage{xunicode}
\usepackage{xltxtra}
\usepackage{polyglossia}
\setdefaultlanguage{latvian}
\usepackage{fixlatvian}
\usepackage{caption}
\usepackage{amsmath}
\usepackage{algorithm}
\usepackage[noend]{algpseudocode}
\usepackage{listings}
\usepackage{tocloft}
\usepackage{float}

\makeatletter
\def\BState{\State\hskip-\ALG@thistlm}
\makeatother
\newcommand\var{\texttt}
\interfootnotelinepenalty=10000
\setotherlanguages{english,russian,french}
\setmainfont[Mapping=tex-text]{Times New Roman}%{LMRoman10}
% Fonts krievu valodai, kurā ir arī krievu valodas burti
\newfontfamily\russianfont{Times New Roman}
% Šos fontus tālāk izmantos chapter virsrakstos un url'os (lai būtu kirilicas burti)
\newfontfamily\sffamily{Verdana}
\captionsetup{justification=centering}
\usepackage{setspace}
\newcommand{\nocontentsline}[3]{}
\newcommand{\tocless}[2]{\bgroup\let\addcontentsline=\nocontentsline#1{#2}\egroup}
% lai varam normāli rakstīt apakšvītras
\usepackage{underscore}
% Lai varam iekļaut attēlus
\usepackage{graphicx}
% Kurā vietā tiks meklēti attēli - relatīvais ceļs attiecībā pret dokumentu
\graphicspath{{./PNG/}{./images/internet/}{./images/self-generated/}}
% Ar šiem PDF'ā būs saliktas saites un tām va uzlikt krāsu
\usepackage{hyperref}
\hypersetup{ colorlinks, citecolor=black, filecolor=black,linkcolor=black,urlcolor=black }

\usepackage{amsmath}
\usepackage{amsfonts}
\usepackage{lipsum} %Lai ģenerētu nejaušus tekstus...
\usepackage{listingsutf8}
\usepackage{xcolor}

%\usepackage{inconsolata}
\lstset{
    language=bash, %% Troque para PHP, C, Java, etc... bash é o padrão
    basicstyle=\small,
    numberstyle=\small,
    numbers=left,
    backgroundcolor=\color{gray!10},
    frame=single,
    tabsize=2,
    rulecolor=\color{black!30},
    title=\lstname,
    escapeinside={\%*}{*)},
    breaklines=true,
    breakatwhitespace=true,
    framextopmargin=1pt,
    framexbottommargin=1pt,
    extendedchars=false,
    inputencoding=utf8
}

\usepackage{titlesec}
\titleformat{\chapter}{\huge\centering\sffamily}{\thechapter}{1pc}{}

\addto\captionslatvian{
\renewcommand\bibname{Izmantotās literatūras un avotu saraksts}
}
\clubpenalty10000
\widowpenalty10000
\setlength{\topmargin}{0cm}
\setlength{\headheight}{0in}
\setlength{\headsep}{0in}
\setlength{\textheight}{22.7cm}
\setlength{\textwidth}{15cm}
\setlength{\oddsidemargin}{0.5in}
\setlength{\evensidemargin}{0.5in}
\usepackage{indentfirst}
\hyphenpenalty=5000
\usepackage{totcount}
\newcounter{nofappendices}
\setcounter{nofappendices}{0}
\regtotcounter{nofappendices}
\newtotcounter{fignum}
\def\oldfigure{} \let\oldfigure=\figure
\def\figure{\stepcounter{fignum}\oldfigure}
\newtotcounter{citnum}
\def\oldbibitem{} \let\oldbibitem=\bibitem
\def\bibitem{\stepcounter{citnum}\oldbibitem}
%% Sarakstam visus mainīgos
%% Mainīgie titullapai, defAutors tiek izmantots arī galvojumā
\def\defAutors{Edgars Skore}
\def\defAugstskola{Ventspils Augstskola}{\fontfamily{russianfont}
\def\defFakultate{Informācijas tehnoloģiju fakultāte}
\def\defSProgrammas{dabas zinātņu maģistra studiju programmas\\
	      datorzinātnēs}
\def\defStudents{2. kursa students \\
	      \defAutors}
\def\defMatrikulasNr{16080002}
\def\defDarbaNosaukums{Cilvēku plūsmas analīze, pielietojot konvolūcijas tīklus}
\def\defDarbaNosaukumsEN{Human flow analysis using Convolutional Neural Networks}
\def\defDarbaNosaukumsFR{La grande question sur la vie, l'univers et le reste}
\def\defDarbaNosaukumsLV{Pagātnes, tagadnes un nākotnes aspekti atbildei uz galveno jautājumu}
\def\defDarbaNosaukumsRU{Ответ на главный вопрос жизни, вселенной и всего такого}
\def\defDarbaVeids{Maģistra darbs}
\def\defFakultatesDekans{doc.  Dr.phys. M.~Ēlerts}
\def\defZinVaditajs{Dr.sc.comp. Gundars Bergmanis-Korāts}
\def\defGads{2018}

\usepackage{lipsum}

\def\abstract{

\vspace*{-4\baselineskip}
	\chapter*{\begin{center} \abstractname \end{center} } % start chapter
	\vspace*{-2.5\baselineskip}
  \addcontentsline{toc}{chapter}{\abstractname} % table of contents line
  \markboth{\MakeUppercase{\abstractname}}{} % header mark
  \thispagestyle{empty}
}
\def\endabstract{}%\clearpage
\newenvironment{absolutelynopagebreak}
{\par\nobreak\vfil\penalty0\vfilneg
	\vtop\bgroup}
{\par\xdef\tpd{\the\prevdepth}\egroup
	\prevdepth=\tpd}
\begin{document}

%%%% Titullapas sākums
\begin{titlepage}
\begin{center}
\textsc{
\defAugstskola\\
\defFakultate}\\
\vspace{2em}
\textbf{\defDarbaVeids}\\
\vspace{2em}
{\LARGE \textbf{\defDarbaNosaukums}}\\
\vspace{2em}
\begin{tabular}{@{}r@{}l@{}}
\parbox[c]{0.4\textwidth}{Autors:}&
\parbox[t]{0.6\textwidth}{
\defAugstskola s\\
\defFakultate s\\
\defSProgrammas\\
\defStudents \\
Matrikulas~Nr. \defMatrikulasNr\vspace{0.7em}\\
\mbox{}\hrulefill\vspace{-0.4em}\\
{\scriptsize(paraksts)}\vspace{2em}} \\
\parbox[c]{0.4\textwidth}{Fakultātes dekāns:}&
\parbox[t]{0.6\textwidth}{
\defFakultatesDekans\vspace{.7em}\\
\mbox{}\hrulefill\vspace{-0.4em}\\
{\scriptsize(paraksts)}\vspace{2em}} \\
\parbox[c]{0.4\textwidth}{Zinātniskais vadītājs:}&
\parbox[t]{0.6\textwidth}{
\defZinVaditajs\vspace{.7em}\\
\mbox{}\hrulefill\vspace{-0.4em}\\
{\scriptsize(paraksts)}\vspace{2em}} \\
\parbox[c]{0.4\textwidth}{Recenzents:} & \vspace{.7em}\\
\multicolumn{2}{@{}c@{}}{
\mbox{}\hrulefill
}\vspace{-0.4em}\\
\multicolumn{2}{@{}l@{}}{
{\scriptsize(Ieņemamais amats, zinātn. nosaukums,
vārds, uzvārds)}
}\vspace{.7em}\\
&\mbox{}\hrulefill\vspace{-0.4em}\\
&{\scriptsize(paraksts)}\\
\end{tabular}
\vfill
Ventspils, \defGads
\end{center}
\end{titlepage}

\onehalfspace
\begin{absolutelynopagebreak}
{
	\selectlanguage{latvian}
	\begin{abstract}
		
		\begin{tabular}{@{}r@{}l@{}}
			\parbox[c]{0.3\textwidth}{\textbf{Darba nosaukums:}}&
			\parbox[t]{0.65\textwidth}{\defDarbaNosaukums} \\
			\parbox[c]{0.3\textwidth}{\textbf{Darba autors:}}&
			\parbox[t]{0.65\textwidth}{\defAutors} \\
			\parbox[c]{0.3\textwidth}{\textbf{Zinātniskais vadītājs:}}&
			\parbox[t]{0.65\textwidth}{\defZinVaditajs} \\
			\parbox[c]{0.3\textwidth}{\textbf{Darba apjoms:}}&
			\parbox[t]{0.65\textwidth}{\textcolor{black}{\pageref{LastPage}} lapas, 4 tabulas,  \total{fignum}~attēli, 5 vienādojumi, \total{citnum}~literatūras avoti, 4 pielikumi} \\
			\parbox[c]{0.3\textwidth}{\textbf{Atslēgas vārdi:}}&
			\parbox[t]{0.65\textwidth}{ Datorredze, Dziļā mašīnmācīšanās, Konvolūcijas neironu tīkli, Objektu detektēšana, Objektu sekošana} \\
			&\\
		\end{tabular}

Cilvēku plūsmas analīze ir metožu kopums, kas sniedz iespēju no video datiem izgūt informāciju par cilvēku pārvietošanos konkrētā vietā. Pētniecības darba mērķis ir izstrādāt risinājumu, kas ļauj veikt cilvēku plūsmas analīzi, izmantojot konvolūcijas tīklu cilvēku noteikšanai un sekošanas algoritmus cilvēku izsekošanai video. Pirms risinājuma izveidošanas tika apskatītas eksistējošās plūsmas analīzes metodes, apskatīti un salīdzināti dažādas konvolūcijas neironu tīklu arhitektūras, objektu detektēšanas algoritmi un sekošanas algoritmi.

Plūsmas analīzes algoritma implementācija tika veidota programmēšanas valodā \textit{Python}, izmantojot \textit{caffe} un \textit{darknet} ietvarus. Tika salīdzinātas abu minēto ietvaru piedāvātās \textit{SSD} un \textit{YOLO} objektu detektēšanas algoritmu implementācijas. Tika implementēti objektu sekošanas algoritmi, balstoties uz minēto detektēšanas sistēmu atgrieztajiem rezultātiem.

Pēc plūsmas analīzes risinājuma izstrādes un rezultātu izvērtēšanas, var secināt, ka, praktiski implementētie sekošanas algoritmi nespēj atrast objektus, kad tie pazūd aiz priekšplānā esošajiem objektiem. Izmantojot video fragmentus, kas filmēti no paaugstināta skatu punkta, tādējādi samazinot iespēju objektiem pārklāties, darba ietvaros izstrādātais risinājums veiksmīgi darbotos.
	\end{abstract}
}
\end{absolutelynopagebreak}
%%%% 1.5 līiniju atstarpe starp rindām
\onehalfspace
{
\selectlanguage{english}
\begin{abstract}

\begin{tabular}{@{}r@{}l@{}}
\parbox[c]{0.3\textwidth}{\textbf{The title:}}&
\parbox[t]{0.65\textwidth}{\defDarbaNosaukumsEN} \\
\parbox[c]{0.3\textwidth}{\textbf{Author:}}&
\parbox[t]{0.65\textwidth}{\defAutors} \\
\parbox[c]{0.3\textwidth}{\textbf{Academic Advisor:}}&
\parbox[t]{0.65\textwidth}{\defZinVaditajs} \\
\parbox[c]{0.3\textwidth}{\textbf{The volume of the work:}}&
\parbox[t]{0.65\textwidth}{\textcolor{black}{\pageref{LastPage}} pages, 4 tables,  \total{fignum}~images, 5 equations, \total{citnum}~literature sources, 4 appendices} \\
\parbox[c]{0.3\textwidth}{\textbf{Keywords:}}&
\parbox[t]{0.65\textwidth}{ Computer vision, Deep learning, Convolutional neural networks, Object detection, Object tracking} \\
&\\
\end{tabular}
%\total{nofimages} % ja nu gadiijumaa vajag custom counter

Human flow analysis is a set of methods that, by using video processing, allows us to make conclusions of people movement. The aim of the paper was to create a solution that would allow analyzing human flow by using convolutional network for human detection and tracking algorithms for human tracking. Before coming up with the solution, the Author observed existing methods of the flow analysis, examined and compared various convolutional neural network architectures, object detection algorithms and the tracking algorithms.

The implementation of the people flow algorithm was made in the programming language, \textit{Python}, using \textit{caffe} and \textit{darknet} frameworks. The author compared implementations of object detecting, offered by both \textit{SSD} and \textit{YOLO} frameworks. Object tracking algorithms were implemented based on the returned results of the already mentioned detection systems.

After the development of the flow analysis algorithm and evaluation of the results the Author concludes that practically implemented tracking algorithms face difficulties when tracking objects that go into the background behind some other objects in the foreground. The developed solution would be successful if video fragments filmed from elevated point of view were used to minimize the possibility of object occlusion.

\end{abstract}
}
\newpage
\setlength\cftparskip{2pt}
\setlength\cftbeforechapskip{2pt}
\tableofcontents
\onehalfspace
\chapter{Ievads}
Ievadā ir jāietver:

\begin{itemize}
\item temata aktualitātes pamatojums;
\item darba mērķis;
\item darba mērķa sasniegšanai veicamo uzdevumu formulējums;
\item izmantojamo pētīšanas metožu un paņēmienu uzskaitījums;
\item literatūras un avotu grupu uzskaitījums (piemēram, speciālā ekonomiskā literatūra, valsts statistikas dati, nepublicētie materiāli no uzņēmuma arhīva u.c.);
\item darba struktūras apraksts;
\item pētījuma temata un perioda norobežojums (ja tas nepieciešams).
\end{itemize}

Mašīnmācīšanās algoritmi ir kļuvuši par neatņemamu sastāvdaļu programmatūras izstrādātājiem un kompānijām, kas savas aplikācijas grib padarīt "gudras". Lai arī cik, mūsdienās, šis jēdziens ir kļuvis populārs, oficiāla mašīnmācīšanās definīcija nav noteikta. Visvienkāršākais mašīnmācīšanās pielietojums ir apstrādāt datus, mācīties no tiem un no iegūtajiem rezultātiem pieņemt lēmumus vai veikt minējumus reālās pasaules problēmu risināšanai. Tā vietā, lai manuāli veidotu programmatūras risinājumus, kas veic kādu uzdevumu, tiek apmācīti datori vai citas ierīces, izmantojot lielus datu apjomus. 
\section{Mašīnmācīšanās pamatjēdziens}
Pasaulē ir daudz un dažādi mašīnmācīšanās algoritmi un katru dienu tiek publicēti simtiem jaunu algoritmu. Tos var sagrupēt pēc apmācības veida (vadītā apmācība (\textit{supervised learning}), nevadītā apmācība (\textit{unsupervised learning}), pusvadītā apmācība (\textit{semi-supervised learning})) kā arī pēc formas vai funkcijas līdzībām (klasifikācija, regresija, lēmumu koki (\textit{decision trees}), klasterēšana (\textit{clustering}), dziļā mašīnmācīšanās (\textit{deep learning})). Neatkarīgi no apmācības veida vai pielietojuma, visas mašīnmācīšanās algoritmu kombinācijas sastāv no klasifikatoriem (atbalsta vektora mašīna, lēmuma koki, neironu tīkli), vērtēšanas funkcijām (varbūtības funkcijas, robežfunkcijas, izmaksu funkcija) un optimizācijas funkcijām (mantkārīgā meklēšana, nepārtrauktās optimizācijas metodes (\textit{continuous optimization})). Izmantojot šīs sastāvdaļas, mašīnmācīšanās algoritmu pamata mērķis ir būt spējīgam funkcionēt ne tikai ar apmācībā piedāvātajiem datiem, bet arī spēt darboties ar datiem, ar kuriem algoritms nav saskāries. 

Atkarībā no veicamā uzdevuma, ir dažādi veidi kā panākt, lai datori vai jebkura cita ierīce mācās, sākot ar visparastākajiem lēmumu kokiem, beidzot ar ģenētiskajiem algoritmiem un mākslīgajiem neironu tīkliem. Lēmumu koki atspoguļo reālās dzīves koka struktūru un tos izmanto gan klasifikācijas, gan regresijas problēmu risināšanā. Analizējot datus, lēmumu kokus var izmantot, lai vizuāli aprakstītu lēmumus un lēmumu pieņemšanu. Mašīnmācīšanās gadījumā, lēmumu kokus apraksta kā klasifikācijas kokus vai regresijas kokus (\textit{CART - Classification and Regression Trees}), atkarībā no veicamā uzdevuma.  Galvenā šo koku doma ir audzēt zarus, pieņemot lēmumus, kuras koka īpašības izvēlēties, zinot apstāšanās nosacījumu \cite{dectree}. Lai gan lēmumu koki nav vispopulārākais mašīnmācīšanās veids, taču tas ir pielietots klasifikācijas problēmu risināšanai dotajos pētījumos \cite{dectreepaper}\cite{pal2003assessment}.

Ģenētiskie algoritmi ir vēl viens algoritmu veids, kuru ir vērts izcelt. Tie ir algoritmi, kuri tiek izveidoti, balstoties uz notikumiem, kurus novēro dabīgajos evolūcijas procesos. Datorzinātnē ģenētiskie algoritmi ir optimizācijas algoritmi, kuri prot patstāvīgi apgūt jaunu informāciju, balstoties uz evolūcijas jēdzieniem kā dabīgā atlase un ģenētika. Ģenētisko algoritmu pamatideja ir simulēt Čārlza Dārvina piedāvāto teorēmu "izdzīvo stiprākais". Risinot problēmu, ģenētiskais algoritms saglabā tikai spēcīgākos indivīdus katrā paaudzē. Šie indivīdi sacenšas par resursiem un iespēju veidot nākamo paaudzi. Jaunās paaudzes tiek veidotas izvēloties vecākus no iepriekšējās paaudzes, veicot \textit{crossover} operāciju un mutāciju. Spēcīgākie indivīdi katrā paaudzē izveidos vairāk pēcnācēju nekā vājie indivīdi, tādējādi katra nākamā indivīdu paaudze kļūs labāka, galu galā iegūstot labāko rezultātu problēmas risināšanai. Beigu nosacījumu nosaka pirms algoritma izpildes, parasti, tiek noteikts paaudžu skaits vai kāds labāko indivīdu rezultātu slieksnis, kuru pēcnācēju paaudze pārsniedz. Ģenētiskie algoritmi gan nespēj risināt klasifikācijas problēmas, bet tos var lietot kā optimizācijas funkciju \cite{genopti} vai kā kārtošanas algoritmu \cite{deb2000fast}. Lai rēķinātu neironu tīklu svarus, var izmantot ģenētiskos algoritmus. \cite{genalg}
\begin{figure}[h]%
	\centering
	\includegraphics[height=6cm]{images/gen-algo-bilde.png} %
	\caption{Ģenētisko algoritmu modelis}%
	\label{fig:example}%
\end{figure}

Mākslīgie neironu tīkli ir viena no populārākajām mašīnmācīšanās izmantotajām metodēm. Tas ir neapstrādāts elektronisks modelis, kas balstīts uz smadzeņu bioloģisko neironu tīklu. Var teikt, ka šāda neironu tīkla modelis, līdzīgi kā smadzenes, mācās no pieredzes. Teorētiski, šādu smadzeņu modelēšana, paredz, ka šāds mašīnmācīšanās risinājums, neprasa dziļas tehniskas zināšanas bioloģijā vai datorzinātnē, bet ir jāspēj tīklu izveidot pareizi kopā saliekot vairākas slāņu kārtas. Šādas, bioloģijas iedvesmotas metodes uzskata par nākamo lielo soli datorzinātnes industrijā.\cite{staff} \par
Iedziļinoties mākslīgo neironu tīklu uzbūvē, neironu tīkls sastāv no daudz, savstarpēji savienotiem mezgliem, kur katrs no mezgliem veic kādu matemātisku operāciju. To, ko atgriež katrs mezgls, nosaka matemātiskā operācija, ko šis mezgls veic kā arī citi parametri, kas specifiski šim mezglam. Šie mezgli galu galā tiek grupēti un šos mezglu grupējumus sauc par slāņiem (no ang. val. - \textit{layer}). \par
Mākslīgie neironu tīkli satur sava veida "mācīšanās likumus", kas ir process, kad tiek mainīti mezglu savienojumu svari atkarībā no informācijas ievadē. 
Kad neironu tīkls ir apmācīts tik tālu, ka lietotājs ir apmierināts, tad tīklam var sākt piedāvāt datus, kuri tad iziet cauri visiem slāņiem, tādā veidā turpinot mācības uz sākumā izveidotā modeļa bāzes. Neironu tīklus ir arī iespējams pārtrenēt, kas nozīmē, ka tīkls atpazīst tikai vienu ienākošo datu tipu. Ja tā notiek, tad mācīšanās vairs nav iespējama. Mākslīgos neironu tīklus izmanto dažādu problēmu risināšanai, piemēram, rakstu zīmju atpazīšanai \cite{nnchars}, attēlu kompresēšanai \cite{dony1995neural} vai pat akciju tirgus analizēšanai \cite{kimoto1990stock}. Šī darba nolūkos autors neironu tīklus izmantos dziļajai apmācībai.
\begin{figure}[h]%
	\centering
	\includegraphics[height=5cm]{images/neironutikls.png} %
	\caption{Vienkāršs neironu tīkla modelis}%
	\label{fig:example}%
\end{figure}
\section{Dziļā mašīnmācīšanās}
Mašīnmācīšanās no dziļās mašīnmācīšanās atšķiras ar to, ka mašīnmācīšanās ir sarežģīti izmantot ļoti lielas datu kopas, taču ar dziļās apmācības metodēm tas ir iespējams. Dziļās mašīnmācīšanās metodes atgriež jebkādas vērtības sākot ar skaitliskām vērtībām, beidzot ar elementiem kā attēli, teksts vai skaņa, taču parastās mašīnmācīšanās metodes spēj atgriezt tikai skaitliskas vērtības kā, piemēram, klasifikācijas indeksu vai kādas funkcijas rezultātu. Mašīnmācīšanās izmanto dažādus automatizētus algoritmus, kas iemācās modeļa funkcijas un paredz nākotnes darbības no padotajiem datiem, kamēr dziļā apmācībā izmanto neironu tīklus, kas laiž datus caur daudz apstrādes slāņiem, lai izšķirtu datu īpašības. Lielākā atšķirība, pēc autora domām, starp mašīnmācīšanos un dziļo mašīnmācīšanos ir tajā, ka dziļās mašīnmācīšanās metodes automātiski datos atrod svarīgās īpašības, kamēr mašīnmācīšanās algoritmos šīs īpašības ir manuāli jānorāda. 
\begin{figure}[h]%
	\centering
	\includegraphics[height=7cm]{images/deeplearning.png} %
	\caption{Mašīnmācīšanās un dziļās mašīnmācīšanās salīdzinājums}%
	\label{fig:example}%
\end{figure}
Dziļā mašīnmācīšanās ļauj apmācīt matemātiskus modeļus, kas izveidoti no vairākiem datu apstrādes slāņiem, ar datiem, kas attēloti kā vairāku līmeņu abstrakcija (pētamā objekta galveno īpašību izdalīšana un mazsvarīgu aspektu ignorēšana). Šīs metodes ir uzlabojušas jaunākās tehnoloģijas balss atpazīšanā, objektu atpazīšanā attēlos, objektu detektēšanā. Dziļā mašīnmācīšanās, izmantojot atpakaļdatošanas algoritmus, sarežģītās datu kopu struktūrās meklē kā datoram vai jebkurai citai ierīcei būtu jāmaina iekšējie parametri starp tīklu slāņiem.

Īpašību apmācība ir metožu kopums, kas atļauj ierīcei padot neapstrādātus datus un automātiski iegūt īpašības, kas nepieciešamas, lai veiktu detektēšanu vai klasifikāciju. Dziļās mašīnmācīšanās metodes ir īpašību apmācības metodes ar vairākiem īpašību slāņiem, kurus iegūst apvienojot vienkāršus, taču nelineārus modeļus, kur katrs modelis pārveido īpašību no viena līmeņa uz augstāku, abstraktāku līmeni. Veicot pietiekami daudz šādus pārveidojumus, algoritmiem ir iespējams iemācīties ļoti sarežģītas darbības.\cite{deepnet} Darba ietvaros, objektu detektēšanai tiks izmantots konvolūciju neironu tīkls (\textit{CNN}), kas ir dziļās mašīnmācīšanās tips.

\section{Konvolūcijas neironu tīkli}
Konvolūcijas neironu tīkli (turpmāk \textit{CNN}) ir izveidoti, lai apstrādātu datus, kas ievadei padoti kā vairāki masīvi, piemēram, divdimensiju masīvi, kas satur pikseļu intensitātes vairākos krāsu kanālos (attēls). Vairākus datu veidus vienkāršojot, tos var izteikt kā masīvus: viendimensijas masīvs priekš signāliem vai skaitļu rindām, divdimensiju masīvi attēliem vai audio spektogrammām un trīsdimensiju masīvs video. Vislabāko rezultātu CNN tīklu veidi sasniedz risinot objektu detektēšanas \cite{li2015convolutional}\cite{matsugu2003subject}, segmentācijas \cite{long2015fully} vai klasifikācijas problēmas \cite{classif}\cite{krizhevsky2012imagenet}\cite{jia2014caffe}. Šī darba ietvaros CNN tiks izmantots objektu, precīzāk cilvēka galvu detektēšanai.

\subsection{Tīklu slāņi}
Līdzīgi parastajiem neironu tīkliem arī konvolūciju neironu tīkli ir izveidoti no vairākiem slāņiem. CNN ir sarežģītāka struktūra kā parastam neironu tīklam. Vienkārša konvolūciju neironu tīkla arhitektūra sastāv no vairākiem, secīgi novietotiem slāņiem, kurus var izdalīt pa tipiem: konvolūcijas slānis (no kurienes arī rodas tīkla nosaukums), nelinearitātes slānis jeb aktivizācijas slānis, apvienošanas slānis un pilnīgi savienotais slānis (līdzīgi kāds tiek izmantots parastajos neironu tīklos). 
\subsubsection{Konvolūcijas slānis}
Konvolūcijas slānis ir CNN galvenā sastāvdaļa. Kā jau no nosaukuma var noprast, konvolūcijas slānī tiek veikta konvolūcijas operācija. Tiek izvēlēts filtrs (kernelis), un šis filtrs tiek pārvietots pāri masīvam (kas var būt gan attēls, gan audio, gan video) un katrā pozīcijā veic konvolūcijas operāciju. No konvolūcijas operācijas iegūtās vērtības tiek saskaitītas un rezultātā iegūts viens skaitlis. Kad konvolūcijas operācija tiek veikta visam masīvam, tiek iegūts masīvs, ko sauc par īpašību karti (no angļu val. \textit{feature map}) un jo vairāk filtrus izmanto, jo dziļāks kļūst attēls. Izmantojot 32x32x3 ievades masīvu un 5x5x3 filtru (filtram jābūt tikpat dziļam, cik dziļš ir ievades masīvs, lai būtu iespējams veikt matricu reizinājumu) tiks iegūta 28x28x1 īpašību karte un jo vairāk filtri tiks pielietoti, jo dziļāka būs šī īpašību karte un vairāk īpašības būs iespējams atrast masīvā. 
\begin{figure}[h]%
	\centering
	\includegraphics[height=4cm]{images/ActivationMap.png} %
	\caption{Konvolūcijas operācija pirmajā solī pielietojot 5x5 izmēra filtru}%
	\label{fig:example}%
\end{figure}
\\
~
\\
~
\\
~
\\

Šī īpašību karte satur informāciju par to kur atrodas minētās īpašības un cik labi šīs īpašības iedarbojas ar filtru, tādējādi norādot cik ļoti katrā masīva punktā atrodas ar filtru raksturotais elements. Pirms konvolūcijas operācijas veikšanas, ir nepieciešams izvēlēties trīs lielumus, kas ietekmēs īpašību karti:

\begin{itemize}
	\item \textbf{Dziļums} - atbilst filtru skaitam, kas izvēlēts konvolūcijas operācijas veikšanai. Ja ar masīvu tiks veikta konvolūcijas operācija ar trīs dažādiem filtriem, tad tiks atgrieztas trīs dažādas īpašību kartes, kuras būs apvienotas vienā trīs dimensiju masīvā ar dziļumu trīs.
	\item \textbf{Solis} - norāda par cik pikseļiem ievades masīvā pārvietosies filtra masīvs. Piemēram, ja konvolūcijas operāciju veic attēlam un solis ir divi, tad filtri neiet cauri katram pikselim, bet katram otrajam sākot no attēla augšējā kreisā stūra. Jo lielāks solis, jo mazākas būs īpašību kartes, kas ir noderīgi, ja tiek izmantoti lieli vai arī daudz filtru. Tādējādi, ir iespējams ietaupīt skaitļošanas resursus. 
	\item \textbf{Nulles-apdare} (no angļu val. \textit{zero-padding}) - tiek izmantota, ievades matricai pieliekot nulles vērtības ap robežām, lai šiem robežu elementiem būtu pilnībā iespējams pielietot filtrus. Nulles-apdares labās īpašības var izmantot arī lai mainītu īpašību kartes izmērus. Izmantojot nulles-apdari, šis process tiek dēvēts par plato konvolūciju (no angļu val. \textit{wide convolution}), bet nulles-apdares neizmantošanu sauc par šauro konvolūciju (no angļu val. \textit{narrow convolution}). 
	\begin{figure}[h]%
		\centering
		\includegraphics[height=2.5cm]{images/zero-padding.png} %
		\caption{Nulles-apdares piemērs \cite{zerpad}}%
		\label{fig:example}%
	\end{figure}
\end{itemize}
\subsubsection{Nelinearitātes slānis}
Nelinearitātes slānis konvolūciju neironu tīklos satur aktivizācijas funkciju, kas ņem no konvolūcijas operācijas atgriezto īpašību karti un izveido aktivizācijas karti. Aktivizācijas funkcija veic matricas elementu reizinājumu izmantojot saņemtos datus, kas nozīmē, ka tiek izvadīts tik pat liels masīvs, cik bija saņemts ievadē. Aktivizācijas funkciju tradicionāli implementē kā sigmoīdu vai hiperbolisku tangensa funkciju un tās galvenais mērķis ir novērst linearitāti. Nesenāki pētījumi gan norāda, ka konvolūcijas neironu tīklos rektificētas lineārās vienības (no angļu val. \textit{rectified linear units}) (ReLUs) strādā labāk kā tradicionālās aktivizācijas funkcijas \cite{nair2010rectified}. 
\begin{figure}[h]%
	\centering
	\includegraphics[height=4cm]{images/sigmoidhiperbol.png} %
	\caption{Sigmoīds un hiperboliskā tangensa funkcija ir populāri\\ izmantotas aktivizācijas funkcijas konvolūcijas neironu tīklos}%
	\label{fig:example}%
\end{figure}

Rektificētās lineārās vienības (ReLUs) ir speciāla implementācija, kas apvieno nelinearitātes un rektifikācijas slāņu (rektifikācijas slānis atgriež ievades datu elementu moduli) operācijas konvolūcijas neironu tīklos. Rektificēta lineāra vienība ir gabalveida (no angļu val. \textit{piecewise}) lineāra funkcija, kas definēta sekojoši:
\[ Y_i^{(l)} = max(0,Y_i^{(l-1)}) \]
\begin{figure}[h]%
	\centering
	\includegraphics[height=4cm]{images/relu.png} %
	\caption{Rektificēta lineāra vienība}%
	\label{fig:example}%
\end{figure}

Konvolūcijas neironu tīklos, ReLUs ir trīs nozīmīgas priekšrocības pār tradicionālajām loģistikas funkcijām (sigmoīds) vai hiperboliskajām tangensa aktivizācijas funkcijām:
\begin{itemize}
	\item ReLUs efektīvi izplata gradientu, kas samazina iespēju saskarties ar gradienta izzušanas problēmu, kas ir parasta problēma dziļajās neironu tīklu arhitektūrās \cite{hochreiter1998vanishing}.
	\item Īpašības karšu negatīvās vērtības, rektifikācijas lineārā vienība pārveido par nullēm, tādējādi tiekot galā ar anulēšanas problēmu (no angļu val. \textit{cancellation problem}) kā arī izvadē iegūtā aktivizācijas karte saturēs daudz izsētāku vērtību apjomu. Šāds vērtību izkaisījums nodrošina stabilitāti gadījumā, ja ievadē notiek nelielas izmaiņas, piemēram, troksnis\cite{glorot2011deep}. 
	\item Ņemot vērā skaitļošanas sarežģītību, ReLUs sastāv no ļoti vienkāršām operācijām, kas nozīmē, ka šīs operācijas smagi neietekmē CNN veiktspēju, kas padara to implementēšanu konvolūcijas neironu tīklos ļoti efektīvu.
\end{itemize}
Šo priekšrocību dēļ, lielākā daļa jaunāko konvolūcijas neironu tīklu arhitektūru, piemēram, \cite{krizhevsky2012imagenet}\cite{DBLP:journals/corr/HeZR015}\cite{simonyan2014very}, izmanto tieši rektificētās lineārās vienības kā aktivizācijas funkciju nelinearitātes slānī.
\subsubsection{Apvienošanas slānis}
Apvienošanas slāņa (no angļu val. \textit{pooling layer}) galvenā funkcija ir samazināt aktivizāciju karšu dimensiju skaitu. Lai samazinātu iespēju pārlieku apmācīt tīklus (no angļu val. \textit{overfitting}) un samazinātu nepieciešamos skaitļošanas resursus, visbiežāk, šis slānis tiek pielietots pēc vairāk citu slāņu operāciju veikšanas (piemēram, pēc vairākiem konvolūcijas un aktivizācijas slāņiem). Apvienošanas operācijas mērķis ir saglabāt jau atrastās īpašības mazākā attēlojumā. Šis mērķis tiek sasniegts atmetot mazsvarīgos datus, lai iegūtu labāku telpisko izšķirtspēju. 

Apvienošanas slānī tiek definēts filtrs, kurš katrā operācijas solī tiks reducēts līdz vienai vērtībai. Līdzīgi kā konvolūcijas slānī, tiek izvēlēts solis pēc cik pozīcijām atkal tiks pielietots apvienošanas filtrs. Kad filtrs ir izmantots visām iespējamajām masīva pozīcijām, izvadē tiek iegūta telpiski samazināta aktivizācijas karte.

Visbiežāk izmantotās redukcijas metodes ir maksimumu apvienošana vai vidējās vērtības apvienošana. Maksimuma apvienošanas filtri meklē vislielāko vērtību filtra reģionā un atmet pārējās vērtības. Vidējās vērtības apvienošanas filtri saglabā filtr reģiona vidējo vērtību. Maksimuma apvienošana demonstrē spēju ātrāk konverģēt salīdzinājumā ar vidējās vērtības apvienošanu un citām metodēm, tāpēc to izmanto visbiežāk \cite{scherer2010evaluation}. 
\begin{figure}[h]%
	\centering
	\includegraphics[height=3cm]{images/maxpool.png} %
	\caption{Vienkāršs maksimuma apvienošanas slāņa modelis \cite{maxpool}}%
	\label{fig:example}%
\end{figure}
\subsubsection{Pilnīgi savienotais slānis}
Pilnīgās savienošanas slāņa (no angļu val. \textit{fully connected layer}) galvenais uzdevums ir, izmantojot no iepriekšējiem slāņiem iegūto īpašību karti, sadalīt ievades datu masīvu pa klasēm, kas balstītas uz apmācībā izmantoto datu kopu. Papildu klasifikācijas funkcijai, pilnīgi savienotos slāņus ir vērts ievest tīklā, lai noskaidrotu īpašību karšu nelineārās kombinācijas. Šis slānis kā ievades datus saņem konvolūcijas, aktivizācijas vai apvienošanas slāņu izvadi un atgriež \textit{N} dimensiju vektoru, kur \textit{N} ir neironu tīklam piedāvāto klašu skaits. Šis klašu jeb atšķirīgu detektējamo objektu skaits ir jādefinē pirms tīkla apmācības. Iepriekš minētais vektors saturēs informāciju par to cik ļoti katrs aktivizācijas kartes elements korelē ar definētajām klasēm. Piemēram, ja tīklam ievadē ir padots attēls ar cilvēku, aktivizācijas kartē būs augstas vērtības īpašībām, kas raksturo, ka objektam ir divas rokas un divas kājas. Summējot no pilnīgi savienotā slāņa izvadē iegūtās varbūtības ir jāiegūst vērtība 1. To nodrošina \textit{Softmax} funkcija, ko izmanto kā aktivizācijas funkciju pilnīgi savienotajā slānī. \textit{Softmax} funkcijai kā ievades datus padod vektoru ar patvaļīgiem rezultātiem, kas iegūti pēc klasifikācijas un šī funkcija saspiež šīs vērtības vektorā, kur visas vērtības ir robežās no 0 līdz 1 un visa vektoru vērtību summa ir 1. \textit{Softmax} aktivizācijas funkcija ir definēta sekojoši:
\[f_j(z) = \frac{e^{z_j}}{\sum_k e^{z_k}} \]
Kur \textit{z} ir masīvs ar vērtībām, kas iegūtas pēc klasifikācijas.
\subsection{Tīklu arhitektūras}
Nav noteikts viens veids kādai ir jābūt pareizai konvolūcijas neironu tīkla struktūrai, taču ir pieņemts, ka konvolūcijas neironu tīkls vienmēr satur jau minētos tīkla slāņus. Mūsdienās, standarta CNN arhitektūru sāk aizvietot ar sarežģītākām tīklu arhitektūrām, kuras visbiežāk parastos četrus konvolūcijas neironu tīklu slāņus maina vietām un atkārto vairākas reizes. Izveidot pašam savu CNN struktūru nav sarežģīti, to pat var padarīt funkcionālu, taču, lai tas funkcionētu labāk kā jau esošās arhitektūras, ir nepieciešams ieguldīt daudz laika testēšanai un pētīšanai. Dotajā brīdī, ir pieejami tūkstošiem, un katru dienu parādās jaunas, konvolūcijas neironu tīklu arhitektūras, šī darba ietvaros ir vērts pieminēt dotās tīklu arhitektūras: \textit{AlexNet}, \textit{VGG Net}, \textit{GooGleNet} un \textit{ResNet}.

\subsubsection{AlexNet}

\textit{AlexNet} tīklu arhitektūru var uzskatīt par pirmo konvolūcijas neironu tīklu arhitektūru, kas uzrādija labus rezultātus, risinot objektu detektēšanas, klasifikācijas un lokalizācijas problēmas \cite{ILSVRC15}. Tā tika izveidota 2012. gadā un ir aprakstīta rakstā \textit{"ImageNet Classification with Deep Convolutional Networks"}\cite{krizhevsky2012imagenet}. 2012. gada ILSVRC (\textit{ImageNet Large-Scale Visual Recognition Challenge}) \textit{AlexNet} sacensībās uzrādīja tik labus rezultātus kā neviena CNN arhitektūra līdz šim \cite{ILSVRC15} un kopš šīm sacensībām CNN vispārīgi sāka gūt popularitāti objektu detektēšanā. 

Salīdzinājumā ar mūsdienu arhitektūrām, \textit{AlexNet} struktūra ir salīdzinoši vienkārša. Tīklā ir 5 konvolūcijas, apvienošanas slāņi un nelinearitātes slāņi un 3 pilnīgi savienotie slāņi. Nelinearitātes slānī tiek izmantota \textit{ReLU} aktivizācijas metode, jo testēšanas procesā, \textit{ReLU} apmācības laiki bija vairākas reizes mazāki, kā ierastajai hiperboliskajai tangensa aktivizācijas funkcijai. 
\begin{figure}[h]%
	\centering
	\includegraphics[height=4cm]{images/alexnet.jpg} %
	\caption{\textit{AlexNet} tīkla arhitektūra \cite{alexnetimage}}%
	\label{fig:example}%
\end{figure}
\subsubsection{VGG Net}
\textit{VGG Net} arhitektūra ir izveidota 2014. gadā un aprakstīta rakstā ar nosaukumu \textit{"Very deep convolutional networks for large-scale image recognition"}\cite{simonyan2014very}. Šīs arhitektūras galvenā īpašība ir vienkāršība un slāņu daudzums (no angļu val. \textit{depth}), saglabājot tīklu nesarežģītu. \textit{VGG Net} ir 19 slāņu konvolūcijas neironu tīkls, kas konvolūcijas slānī izmanto mazus, 3x3 izmēra filtrus ar soli 1 un 1 \textit{zero-padding} jeb nulles-apdares operāciju un apvienošanas slānī izmanto 2x2 filtrus ar soli 2.
Pēc tīkla arhitektūras autoru domām, lai gan 3x3 izmēru filtri ir mazi, divus filtrus ir iespējams apvienot (novietojot 2 konvolūcijas slāņus vienu pēc otra), simulējot 5x5 izmēra filtru un apvienojot 3 konvolūcijas filtrus tiks simulēts 7x7 filtrs. Vairāku filtru apvienošana dod iespēju iegūt lielāku filtra izmēru, saglabājot labās mazo filtru īpašības, piemēram, ar vairākiem konvolūcijas slāņiem, ir iespējams arī vairākas reizes pielietot \textit{ReLU} aktivizācijas funkciju. Izmantojot vairākus mazus filtrus, tīkls var iemācīties smalkākas, sarežģītākas īpašības būtiski neietekmējot skaitļošanas resursu izmantošanu.
\begin{figure}[h]%
	\centering
	\includegraphics[height=7cm]{images/vggnet.png} %
	\caption{\textit{VGG Net} tīklu struktūras \cite{simonyan2014very}}%
	\label{fig:example}%
\end{figure}

\textit{VGG Net} arhitektūrā, pēc katra apvienošanas slāņa, konvolūcijas slāņos izmantotas filtru skaits dubultojas. Kā rezultātā datu telpiskais izmērs pēc konvolūcijas un apvienošanas slāņu operācijām samazinās, bet datu dziļums palielinās, kas ir rezultāts daudz filtru pielietošanai. Attēlā 1.10 ir parādīts, ka \textit{VGG Net} struktūra ir sadalīta blokos un katrā blokā konvolūcijas operācija, ar vienāda izmēra filtriem, tiek veikta vairākas reizes, lai iegūtu smalkākas īpašības.
\subsubsection{GoogLeNet}
Pretēji \textit{VGG Net} principam "vienkāršība un dziļums", \textit{GoogLeNet} arhitektūrai ir ļoti sarežģīta struktūra, kurā ir iebūvēti \textit{Inception} modeļi. \textit{GoogLeNet} 2015. gadā izveidoja kompānijas \textit{Google} pētnieki un aprakstīja tīkla arhitektūru rakstā "\textit{Going Deeper with Convolutions}" \cite{szegedy2015going}. \textit{GoogLeNet} arī uzvarēja 2014. gada ILSVRC sacensībās \cite{ILSVRC15}. Šī tīkla arhitektūra bija pirmā, kas novērsās no standarta metodēm: konvolūciju neironu tīklus kārtot slāņus secīgi, vienu pēc otra, piedāvājot slāņus kārtot paralēli.

\textit{GoogLeNet} arhitektūru sakārtojot secīgi, tā izmantotu milzīgus atmiņas apjomus un prasītu pamatīgus skaitļošanas resursus, taču, izmantojot \textit{Inception} modeļus, šo problēmu var samazināt.
\begin{figure}[h]%
	\centering
	\includegraphics[height=4cm]{images/GoogLeNet2.png} %
	\caption{\textit{GoogLeNet} tīkla struktūra \cite{szegedy2015going}}%
	\label{fig:example}%
\end{figure}

Attēlā 1.11 var novērot, ka tīkla struktūra ir sadalīta moduļos (piemērs zaļā taisnstūrī) un šie moduļi satur blokus, kas ir novietoti paralēli viens otram. Katrs šis modelis ir \textit{Inception} modelis. Šis modulis ir kā vairāki konvolūcijas filtra ievades punkti, kurus apstrādā vienlaicīgi, paralēli, visus rezultātus pēc tam savieno, šādā veidā no katras ievades datu vienības tiek veikta vairāku līmeņu īpašību (\textit{feature}) izgūšana.

\begin{figure}[h]%
	\centering
	\includegraphics[height=4cm]{images/inception.png} %
	\caption{\textit{Inception} modelis \cite{szegedy2015going}}%
	\label{fig:example}%
\end{figure}

\subsubsection{ResNet}
No augstāk minēto arhitektūru aprakstiem, var izdarīt secinājumu, ka palielinot slāņu skaitu, palielināsies tīkla precizitāte (ja netiek pieļauta tīkla pār-apmācīšana (no angļu val. \textit{overfitting})). \textit{Microsoft} komandas, 2015. gadā izveidotā CNN arhitektūra ir 152 slāņu arhitektūra, kas uzstādīja jaunus rekordus klasifikācijā, detektēšanā un lokalizācijā, uzvarot 2015. gada ILSVRC sacensībās \cite{ILSVRC15}. 

Neskaitot lielo slāņu skaitu, \textit{ResNet} tīklu struktūra no pārējām, jau minētajām, arhitektūrām atšķirt tas, ka \textit{ResNet} sastāv no "atlikumu blokiem" (no angļu val. \textit{residual blocks}). Šo bloku galvenā doma ir saglabāt tajā padoto informāciju un pievienot to izejošajai informācijai, nevis pilnībā aizmirst ievades datus kā tas notiek citās arhitektūrās \cite{he2016deep}. 

\begin{figure}[h]%
	\centering
	\includegraphics[height=4cm]{images/resnet.png} %
	\caption{\textit{ResNet} atlikumu bloks \cite{he2016deep}}%
	\label{fig:example}%
\end{figure}
\subsection{Tīklu apmācība}
\chapter{Literatūras apskats}

Ir aprakstītas vairākas metodes, lai veiktu cilvēku plūsmu analīzi attēlos un video fragmentos \cite{brostow2006unsupervised,chen2013cumulative,ge2009marked,chen2015person,lempitsky2010learning}. Vispārīgi, pirmie pētījumi tika vērsti uz detektēšanas veida izvēli un vai problēmu var risināt izmantojot segmentācijas metodes \cite{tu2008unified}. Šīs metodes nelabvēlīgi ietekmēja objektu nostāšanās vienam aiz otra, objektu pazušana un nekārtīgs (pārblīvēts TODO pārbaudīt tulkojumu (high clutter background)) fons attēlos. Jaunākos risinājumus var vispārīgi sadalīt trīs kategorijās: risinājumi, kas balstīti uz regresiju, risinājumi, kas balstīti uz pūļa blīvuma novērtējumu un risinājumi, kas balstīti uz konvolūcijas neironu tīkliem. 

\subsubsection{Risinājumi, kas balstīti uz regresiju}
Lai novērstu objektu paslēpšanos attēlā un pārblīvēta fona problēmas, pētnieki mēģināja skaitīt cilvēkus izmantojot regresiju. Parasti, regresija šādos risinājumos tiek veikta starp dažādām attēla īpašībām un objektu skaitu. Šāds risinājums sadala attēlu vairākos mazākos attēlos un katram šim mazajam attēlam tiek veikta aptuvenā skaita noteikšana izmantojot segmentācijas metodes. Lai noteiktu kopējo cilvēku skaitu attēlā, ir jāsaskaita katra mazā attēla aptuvenie novērtējumi. Lai apmācītu šādu sistēmu, pirmais no attēla apstrādes posmiem ir atmest attēla fonu un veikt \textit{ground-truth} novērtējumu, kas nozīmē, ka tiek manuāli izskaitīts cilvēku skaits katrā sadalītajā attēlā  \cite{chan2009bayesian,ryan2009crowd,chen2012feature}.

Šāda pieeja tika izveidota, pieņemot, ka dotajai cilvēku skaitīšanas sistēmai būtu vieglāk novērtēt cilvēku daudzumu katrā grupā atsevišķi, nevis novērtēt cilvēku daudzumu visam pūlim vienlaikus. Pieņemot, ka attēlā ir pūlis ar 20 cilvēkiem, šo pūli var sadalīt divās lielās grupās vai arī desmit pāros. Ņemot šādu attēlu vispārīgi, šāds pūļa sadalījums var nebūt tik skaidri novērtējams, jo attēlam vispārīgi būtu daudz vairāk atšķirīgu īpašību. Eksistējošas metodes, kas izmanto visu attēlu no attēla iegūst daudz vairāk īpašību, kas nozīmē, ka ir nepieciešami vairāk apmācības datu \cite{chan2008privacy}. Pētot citus rakstus var secināt, ka regresijas metodes parasti sastāv no divām lielām komponentēm: zema līmeņa īpašību iegūšanas un regresijas modeļa implementēšanas \cite{xiong2017spatiotemporal}. 

\subsubsection{Risinājumi, kas balstīti uz pūļa blīvuma novērtējumu}
Lai gan pūļu skaitīšanas risinājumi, kas balstīti uz regresiju veiksmīgi tika galā ar objektu pārklāšanās un pārblīvētā fona problēmām, tika palaista garām svarīga telpiska informācija, jo regresija tika izmantota uz lokālajām īpašībām (katra sadalītā attēla īpašībām atsevišķi). Pētījumā \cite{lempitsky2010learning} tiek piedāvāts jauns veids kā iemācīties lineāras attiecības starp sadalīto attēlu īpašībām un attiecīgās objektu blīvuma kartes izmantojot regresiju. Novērojot, ka iemācīties lineāras attiecības ir sarežģīts uzdevums, tika izveidots risinājums, kas piedāvā iemācīties nelineāras attiecības starp sadalīto attēlu īpašībām un objektu blīvuma kartēm izmantojot nejaušo mežu ietvaru (no angļu val. \textit{random forest framework}) \cite{pham2015count}. Vairāki mūsdienu risinājumi piedāvā metodes, kas balstās uz blīvuma karšu regresiju \cite{wang2016fast,xia2016block}. 

\subsubsection{Risinājumi, kas balstīti uz konvolūcijas neironu tīkliem}
Tā kā mūsdienās klasifikācijā un atpazīšanas uzdevumu risināšanā ļoti veiksmīgi darbojas uz konvolūcijas neironu tīkliem balstītas metodes. Pētnieki ir izveidojuši CNN, ar mērķi veikt pūļa skaitīšanu un blīvuma novērtējumu \cite{wang2015deep,shang2016end,walach2016learning}. Pretēji jau eksistējošajām metodēm, kas pūļa skaitu novērtē izmantojot attēla sadalīšanas metodes, Šangs \cite{shang2016end} piedāvā metodi, kas veic novērtējumu izmantojot CNN. Šī metode novērtējumu veic paralēli skaitot cilvēkus gan globālajā kontekstā, gan lokālajā kontekstā. Žangs \cite{zhang2016single} piedāvāja daudz-kolonnu arhitektūru, kas izgūst īpašības dažādos mērogos. Līdzīgi šai metodei, tika izveidots skaitīšanas modelis, kas veica novērtējumu pūļa blīvuma kartēm, ko nosauca par \textit{Hydra CNN} \cite{onoro2016towards}. Pētnieks Mardsens \cite{marsden2017resnetcrowd} pētīja pilnīgos konvolūciju neironu tīklus un vairākuzdevumu (no ang. val. \textit{multi-task}) apmācību, kuru apvienojot veica cilvēku skaitīšanu. Minētie vairākuzdevumu apmācības un daudz-kolonnu risinājumi ir sasnieguši labus rezultātus, uzrādot salīdzinoši zemu skaitīšanas kļūdu. Balstoties uz minētajiem risinājumiem var izdarīt sekojošus novērojumus \cite{sindagi2017generating}:
\begin{itemize}
	\item Šīs metodes neietver kontekstuālu informāciju, kas ir svarīgi, lai iegūtu labākus rezultātus;
	\item Lai gan eksistējošie risinājumi izmanto regresiju ar pūļu blīvuma kartēm, šie risinājumi ir vairāk balstīti uz skaitīšanas kļūdas samazināšanu, nevis blīvuma karšu kvalitātes uzlabošanu;
\end{itemize}

Šī darba ietvaros tiks veikta skaitīšana pēc detektēšanas. Skaitīšana pēc detektēšanas nozīmē, ka tiks izmantots atsevišķs objektu detektēšanas risinājums, kas lokalizēs pēc konvolūciju neironu tīkla klasifikācijas atrastās klases. Tad, kad lokalizētas visas objekta instances, skaitīšanas uzdevums paliek elementārs. Taču objektu detektēšanas problēmām nav pilnībā atrasti risinājumi, it īpaši, ja vairākas objektu instances attēlā pārklājas.// šite jāturpnia no Learning To Count Objects in Image
\chapter{Plūsmas analīzes algoritma implementācija}
\section{Cilvēku galvas noteikšanas datu sagatavošana}
\section{Konvolūcijas neironu tīkla apmācība}
\section{Sekošanas algoritmu implementācija} 
\chapter{Rezultāti}

Lai gan ierobežotu skaitļošanas resursu dēļ abu darbā apskatīto detektēšanas sistēmu konvolūcijas neironu tīklu apmācība tika veikta ļoti neilgu laiku (aptuveni nedēļu), iegūtie rezultātus var izmantot kā pamatu nopietnākas cilvēku plūsmas analīzes sistēmas izstrādei. Darba izstrādes laikā autoram neizdevās praktiski implementēt \textit{SiamFC} sekošanas algoritmu, kas ir balstīts uz vairākiem konvolūcijas neironu tīkliem un potenciāli varētu krietni uzlabot iegūtos rezultātus. Veidojot rezultātus, tiks salīdzināta objektu sekošana 3 dažādos video fragmentos ar 2 dažādām objektu detektēšanas sistēmām un 2 dažādiem sekošanas algoritmiem. Pēc abu detektēšanas sistēmu apmācības pirmie rezultāti, ko salīdzināt ir tīklu precizitātes metrika \textit{mAP}. \textit{Caffe} ietvarā \textit{SSD} detektēšanas sistēmai precizitāti var aprēķināt ietvara direktorijā izsaucot komandu \textit{"python examples/ssd/score_ssd_pascal.py"}, šī komanda atgrieza vērtību 71\%. \textit{Darknet} ietvarā \textit{YOLO} detektēšanas sistēmai precizitāti var aprēķināt ietvara direktorijā izsaucot komandu\textit{ "./darknet detector map data/obj.data cfg/yolo-obj.cfg backup/yolo-obj_24900.weights"}, šī komanda atgrieza vērtību 75\%. 

Veidojot simulāciju, secīgi, katrs kadrs tika padots detektēšanas sistēmai, kas atgrieza pārliecības novērtējumus. Atgrieztie rezultāti satur atrasto objektu ierobežojošo logu izmērus, atrašanās vietu un pārliecību. Katram video fragmentam pārliecības slieksnis tika noteikts pēc iespējas mazāks, lai, izvairoties no nepareiziem detektēšanas rezultātiem, iegūtu visvairāk detekcijas. No detektēšanas iegūtajiem rezultātiem tika sākta sekošanas operācija, atzīmējot centrus kadros, kur pabijis sekotājs, lai varētu uzzīmēt kustības trajektoriju.

Salīdzinot \textit{MIL} un \textit{AdaBoost} sekošanas algoritmus, \textit{MIL} algoritms strādāja daudz lēnāk un tērēja vairāk skaitļošanas resursu, kamēr \textit{AdaBoost} algoritms bija daudz ātrāks, taču nebija tik precīzs. Pētījumā \cite{lehtola2017evaluation} aprakstīts, ka \textit{AdaBoost} algoritms darbojās līdz 5 kadriem sekundē, kamēr \textit{MIL} algoritms darbojās ar 1 kadru sekundē, šādus rezultātus novēroju arī šī darba ietvaros. 

\paragraph{\textit{SSD} detektēšanas sistēma ar \textit{OpenCV} sekošanas algoritmiem}
\hfill\par

Apmācībai \textit{SSD} detektēšanas sistēmai ievades dati tika padoti izmērā 512x512 pretēji standarta 300x300. Palielinot šo izšķirtspēju tiek upurēts apmācības ātrums, taču iespējams iegūt precīzākus rezultātus. Darba turpinājumā tiks apskatīti \textit{MIL} sekošanas algoritma rezultāti. 
 
\begin{figure}[h]%
	\centering
	\includegraphics[height=6cm]{images/ssd1.png} %
	\caption{\textit{SSD} detektēšanas sistēma ar \textit{MIL} sekošanas algoritmu. Pirmais video.}%
	\label{fig:example}%
\end{figure}
\newpage
Attēlā 4.1. ar zilu krāsu atzīmēts sekošanas algoritmam interesējošais reģions. Ar oranžo krāsu atzīmētais ceļš norāda, ka sieviete attēla labajā pusē ir uztverta uzreiz kā sasniegts interesējošais reģions, taču ar zaļo krāsu atzīmētā sieviete ir uztverta vēlu. Vēlamais rezultāts ar zaļo krāsu atzīmēto sievieti uztvertu uzreiz kā tā parādītos durvīs, šajā gadījumā, detektēšana sievieti atradusi pāris kadrus vēlāk kā vēlams. Ir iespējams konfigurēt detektēšanas sistēmu, lai uztvertu objektus ar zemāku pārliecības slieksni, bet tas var izraisīt kļūdainus rezultātus, piemēram, atzīmēt cilvēku kāju kā galvu.

\begin{figure}[h]%
	\centering
	\includegraphics[height=6cm]{images/ssd2.png} %
	\caption{\textit{SSD} detektēšanas sistēma ar \textit{MIL} sekošanas algoritmu. Pirmais video.}%
	\label{fig:example}%
\end{figure}

Attēlā 4.2. attēlots brīdis pāris sekundes vēlāk, kad sieviete, kas attēlā 4.1. bija atzīmēta ar oranžo krāsu jau pametusi telpu, bet sekotājs apstājies pie durvīm. No tā var secināt, ka objektam pazūdot no sekošanas algoritma redzes loka, sekošanas algoritms pazaudē objektu un vairs nespēj to atrast. \textit{OpenCV} sekošanas programmējamā interfeisā piedāvātās metodes šādā brīdī nenorāda, ka objekts pazudis, bet uzskata, ka joprojām ir atrasts. Ar zaļo krāsu atzīmētajai sievietei, sekošanas algoritms ir veiksmīgi izsekojis līdz attēlā norādītajai atrašanās vietai, jo viņas ceļā nav bijuši šķēršļi. Vērts piebilst, ka dotais video fragments ir ļoti augstas izšķirtspējas, kas ļauj veikt objektu detektēšanu ar augstāku precizitāti. 

\begin{figure}[h]%
	\centering
	\includegraphics[height=6cm]{images/ssd3.png} %
	\caption{\textit{SSD} detektēšanas sistēma ar \textit{MIL} sekošanas algoritmu. Otrais video.}%
	\label{fig:example}%
\end{figure}

Attēlā 4.3. ir attēlots fragments no cita video, kas ir zemākas izšķirtspējas un filmēts citā vidē. Līdz dotajam brīdim, video fragmentā uztverti 4 objekti, kamēr 1 objekts no visiem nav atrasts. Ar oranžo,violeto un zaļo krāsu atzīmētie objekti ir uztverti pāris kadrus pēc tam, kad tie parādījušies interesējošajā reģionā, kamēr ar sarkano krāsu atzīmētais objekts uztverts uzreiz. Arī šajā gadījumā ir iespējams konfigurēt detektēšanas sistēmu, lai uztvertu objektus ar zemāku pārliecības slieksni, bet tas var izraisīt kļūdainus rezultātus, piemēram, atzīmēt cilvēku kāju kā galvu.

\begin{figure}[h]%
	\centering
	\includegraphics[height=6cm]{images/ssd4.png} %
	\caption{\textit{SSD} detektēšanas sistēma ar \textit{MIL} sekošanas algoritmu. Otrais video.}%
	\label{fig:example}%
\end{figure}

Līdz attēlā 4.4. redzamajam brīdim objektiem, kas ir tuvu filmēšanas punktam ir veiksmīgi izsekots, kamēr vīrietis, kas bija atzīmēts ar violeto krāsu vairs netiek uztverts un priekšplānā esošie objekti, šķērsojot to, lika sekošanas algoritmam kļūdīties un violetajam objektam atkal parādoties vairs nespēja tam izsekot. 

\begin{figure}[h]%
	\centering
	\includegraphics[height=6cm]{images/ssd5.png} %
	\caption{\textit{SSD} detektēšanas sistēma ar \textit{MIL} sekošanas algoritmu. Trešais video.}%
	\label{fig:example}%
\end{figure}

Attēlā 4.5. ir attēlots fragments no cita video, kas ir ļoti zemas izšķirtspējas un filmēts citā vidē. Dotais fragments ir pirmie video kadri, attēlā ir redzami daudz cilvēki, taču detektēšanas algoritms ir atradis tikai divus. Par iemeslu var uzskatīt ļoti zemo video fragmenta izšķirtspēju vai apmācības datu nepietiekamību.

\begin{figure}[h]%
	\centering
	\includegraphics[height=6cm]{images/ssd6.png} %
	\caption{\textit{SSD} detektēšanas sistēma ar \textit{MIL} sekošanas algoritmu. Trešais video.}%
	\label{fig:example}%
\end{figure}

Attēlā 4.6. ir attēlots brīdis no tā paša video, pāris sekundes vēlāk, kad rāmī ir parādījušies un uztverti 3 jauni objekti. Detektēšana gan tiem nav notikusi līdz ar parādīšanos interesējošajā reģionā, bet pāris kadrus vēlāk. 

Apskatot sagatavotās sekošanas trajektorijas, var secināt, ka\textit{AdaBoost} sekošanas algoritms strādā gandrīz tik pat labi kā \textit{MIL} algoritms, brīžiem gan novērojami iztrūkumi, sekošanas laikā sekotājs brīžiem pazaudē objektu, bet ātri, pēc pāris kadriem, to atrod. Līdzīgi kā \textit{MIL} algoritmam, arī \textit{AdaBoost} netiek galā ar objektiem, kas video fragmentos novietojas aiz šķēršļiem, bet izpildes laikā ir ātrāks. Ja gala risinājumā nepieciešams izvēlēties starp \textit{MIL} un \textit{AdaBoost}, autors izvēlētos \textit{MIL}, jo, lai gan \textit{AdaBoost} ir nedaudz ātrāks, \textit{MIL} algoritms objektiem seko stabilāk.
\begin{figure}[h]%
	\centering
	\includegraphics[height=6cm]{images/ssd7.png} %
	\caption{\textit{SSD} detektēšanas sistēma ar \textit{AdaBoost} sekošanas algoritmu. Pirmais video.}%
	\label{fig:example}%
\end{figure}
\paragraph{\textit{YOLO} detektēšanas sistēma ar \textit{OpenCV} sekošanas algoritmiem}
\hfill\par
Salīdzinot ar darba izstrādes laikā apmācīto \textit{SSD} objektu detektēšanas sistēmu, \textit{YOLO} detektēšanas sistēma tika apmācīta krietni mazāku laiku, kas arī atspoguļojās rezultātos. Augstas izšķirtspējas nekustīgos attēlos \textit{YOLO} detektēšanas sistēma rādīja labus rezultātus, taču video fragmentos detektēšanas kvalitāte bija zemāka kā \textit{SSD} detektēšanas sistēmai. Vēl viens iemesls salīdzinoši sliktākajai detektēšanas kvalitātei varētu būt praktiskā implementācija. Implementējot detektēšanu \textit{darknet} ietvarā, grūtības sagādāja no video fragmenta iegūto kadru pārveidošana uz \textit{darknet} \textit{Image} objektu. Tāda iemesla dēļ, pirms veikt detektēšanu, video fragmenti tika sadalīti attēlos un detektēšana tika veikta katram attēlam atsevišķi nevis \textit{OpenCV VideoFile} objektam kā \textit{SSD} gadījumā. Tādā veidā, sekošanas algoritms var pazaudēt savienojumus starp video kadriem.
\newpage
\begin{figure}[H]%
	\centering
	\includegraphics[height=5cm]{images/yolo1.png} %
	\caption{\textit{YOLO} detektēšanas sistēma ar \textit{AdaBoost} sekošanas algoritmu. Pirmais video.}%
	\label{fig:example}%
\end{figure}
\begin{figure}[H]%
	\centering
	\includegraphics[height=5cm]{images/yolo2.png} %
	\caption{\textit{YOLO} detektēšanas sistēma ar \textit{AdaBoost} sekošanas algoritmu. Otrais video.}%
	\label{fig:example}%
\end{figure}
\begin{figure}[H]%
	\centering
	\includegraphics[height=5cm]{images/yolo3.png} %
	\caption{\textit{YOLO} detektēšanas sistēma ar \textit{AdaBoost} sekošanas algoritmu. Trešais video.}%
	\label{fig:example}%
\end{figure}
\newpage
Attēlā 4.8. var novērot, ka sekojot rodas iztrūkumi un detektēšana ir darbojusies līdzīgi \textit{SSD} detektēšanas sistēmai. Attēlā 4.9. priekšplānā esošie cilvēki ir detektēti labāk kā \textit{SSD} sistēmai, taču fonā esošie netiek atrasti. Kā jau darbā minēts, \textit{YOLO} slikti detektē mazus objektus, kas varētu būt iemesls. Sekošanas algoritms attēlā 4.9. darbojas līdzīgi kā \textit{MIL} sekošanas algoritms, kur objekti, aizklājot citus, pārtvēra to ierobežojošos logus. 

Tā kā video fragments attēlā 4.10. ir ļoti zemas izšķirtspējas, visi objekti ir salīdzinoši mazi, kas \textit{YOLO} sagādā lielas problēmas. Fona esošais objekts ir detektēts veiksmīgi, bet pārējie cilvēki, kas pārvietojas fonā tiek detektēti ļoti vāji.
\chapter{Secinājumi un priekšlikumi}
\paragraph{Secinājumi}
\hfill\par
\begin{enumerate}
	\item Jaunākie risinājumi balss atpazīšanā, objektu klasifikācijā un objektu detektēšanā ir veidoti izmantojot dziļās mašīnmācīšanās metodes. Šos risinājumus veido izmantojot konvolūcijas neironu tīklus.
	\item Jo vairāk konvolūcijas slāņu (filtru) neironu tīklā, jo labāk šis tīkls veiks klasifikāciju, taču vairāk slāņu izmantošana nozīmē papildu skaitļošanas jaudas nepieciešamību. Izveidot savu konvolūcijas neironu tīklu arhitektūru ir viegli, taču, lai to padarītu labāku par jau eksistējošajām arhitektūrām, ir nepieciešams ieguldīt daudz laika pētījumos un eksperimentos.
	\item Lai efektīvi darbotos ar konvolūciju neironu tīkliem, apmācību ir nepieciešams veikt, izmantojot grafisko procesoru, kam ir daudz vairāk skaitļošanas vienības kā centrālajiem procesoriem.
	\item Dotajā brīdī, \textit{ReLU} ir labākā konvolūciju neironu tīklos izmantotā aktivizācijas funkcija. 
	\item Izvēloties programmatūras ietvarus darbam ar konvolūcijas neironu tīkliem ir pieejami daudz varianti, taču, lai izvēlētos ietvaru ir svarīgs gala risinājuma mērķis un pielietojums. Piemēram, \textit{Caffe} ir speciāli veidots darbam ar konvolūcijas neironu tīkliem un \textit{darknet} satur oriģinālo \textit{YOLO} implementāciju. 
	\item Izmantojot augstas izšķirtspējas attēlus, apmācības laikā, var uzlabot detektēšanas sistēmas veiktspēju, taču tiek upurēti skaitļošanas resursi. Jāņem vērā, ka veicot apmācību ar augstas izšķirtspējas attēliem, bet pēc tam veicot detektēšanu ļoti zemas izšķirtspējas attēliem, tā darbosies vāji.
	\item Objektu sekošanas algoritmus ir nepieciešams implementēt, lai nepazaudētu objektu, kad detektēšanas sistēmas vairs nevar to atrast, kā arī sekošanas operācija tērē mazāk resursu kā detektēšana. No sekošanas algoritmu rezultātiem var analizēt objektu pārvietošanās trajektorijas.
	\item Jo lielāka un daudzveidīgāka apmācības datu kopa, jo dažādākus objektus detektēšanas sistēmas var atrast.
	\item Nepopulāras datu kopas (datu kopas, kas nav \textit{Pascal VOC, MS COCO, ILSVRC}) pielietotu ar populārākajiem konvolūcijas neironu tīklu programmatūras ietvariem, var nākties veikt datu pārveidojumus kā anotāciju formāta maiņu.
	\item \textit{Python} programmēšanas valodā ir pieejamas daudz un dažādas datorredzes un mašīnmācīšanās programmatūras bibliotēkas. Šīs programmatūras bibliotēkas kā arī elementārā koda sintakse padara \textit{Python} programmēšanas valodu viegli izmantojamu, lai darbotos dziļās apmācības laukā.
	\item Lai sasniegtu labākus detektēšanas sistēmu rezultātus, konvolūcijas neironu tīklu apmācība ir jāveic daudz ilgāk kā šī darba ietvaros kā arī jāveic datu uzlabošana (no angļu val \textit{data augmentation}).
	\item \textit{AdaBoost} sekošanas algoritms darbojās ātrāk kā \textit{MIL}, taču \textit{MIL} algoritms uzrādīja labākus rezultātus.
	\item Šī darba ietvaros praktiski implementētie sekošanas algoritmi nespēja atrast objektus, kad tie pazūd fonā aiz priekšplānā esošajiem objektiem.
	\item Ja video fragments ir augstas izšķirtspējas un uzņemts ar kameru no paaugstināta skatu punkta, kur skaitīšanas līniju var novietot horizontāli, cilvēki pārvietojas vertikāli un apmācības dati atbilstu problēmai, tad, šī darba ietvaros izveidotais risinājums, teorētiski, būtu labi funkcionējošs.
\end{enumerate}
\paragraph{Priekšlikumi}
\hfill\par
\begin{enumerate}
	\item Lai uzlabotu šī darba ietvaros izveidoto sistēmu, ir nepieciešams abas izmantotās detektēšanas sistēmas apmācīt ilgāk un veikt datu uzlabošanu (no angļu val. \textit{data augmentation}). Izmantotajā datu kopā ir vairāk kā 220 tūkstoš attēlu un šī darba ietvaros veiktās apmācības ilgums ir nepietiekams, lai iegūtu labākus rezultātus ar attēliem, kas nav no datu kopas.
	\item Lai iegūtu labākus detektēšanas rezultātus un, pieņemot, ka ir pieejami neierobežoti skaitļošanas resursi, šī darba ietvaros \textit{VGG Net} konvolūcijas neironu tīklu arhitektūru būtu vērtīgi aizstāt ar \textit{ResNet} konvolūcijas neironu tīklu arhitektūru. \textit{ResNet} arhitektūra satur daudz vairāk slāņus un labāk detektē objektus, bet apmācība pieprasa augstu skaitļošanas jaudu, ko nenodrošina darbā izmantotā grafiskā karte. 
	\item Lai uzlabotu sekošanas algoritmu veiktspēju, var izmantot uz konvolūcijas neironu tīklu apmācību balstītus sekošanas algoritmus. Darbā minētais \textit{SiamFC} ir viena no šādām sistēmām, kuru gan darba izstrādes laikā neizdevās praktiski implementēt. Papildu detektēšanas sistēmu apmācībai, nākamais solis risinājuma izstrādē būtu pielietot \textit{SiamFC} sekošanas sistēmu. 
\end{enumerate} 
\bibliographystyle{ieeetr}
\selectlanguage{latvian}
\bibliography{src/links,src/articles,src/books}	%to load the *.bib files ../articles,../books,
\addcontentsline{toc}{chapter}{Izmantotās literatūras un avotu saraksts}

\label{LastPage}

%% Te vajadzētu pielikumus
\appendix
\chapter{Pirmais pielikums}
\label{appendix:pielikums1}
\begin{lstlisting}[basicstyle=\tiny]
cur_dir=$(cd $( dirname ${BASH_SOURCE[0]} ) && pwd )
root_dir=$cur_dir/../..
cd $root_dir
redo=1
data_root_dir="$HOME/data/VOCdevkit"
dataset_name="VOC0712"
mapfile="$root_dir/data/$dataset_name/labelmap_voc.prototxt"
anno_type="detection"
db="lmdb"
min_dim=0
max_dim=0
width=0
height=0
extra_cmd="--encode-type=jpg --encoded"
if [ $redo ]
then
extra_cmd="$extra_cmd --redo"
fi
for subset in test trainval
do
python $root_dir/scripts/create_annoset.py --anno-type=$anno_type --label-map-file=$mapfile --min-dim=$min_dim --max-dim=$max_dim --resize-width=$width --resize-height=$height --check-label $extra_cmd $data_root_dir $root_dir/data/$dataset_name/$subset.txt $data_root_dir/$dataset_name/$db/$dataset_name"_"$subset"_"$db examples/$dataset_name
\end{lstlisting}
\addtocounter{nofappendices}{1}
\chapter{Otrais pielikums}
\label{appendix:pielikums2}
\begin{lstlisting}[basicstyle=\tiny]
import xml.etree.ElementTree as ET
import pickle
import os
from os import listdir, getcwd
from os.path import join
sets=[('2007', 'train'), ('2007', 'val'), ('2007', 'test')]
classes = ["head"]
def convert(size, box):
dw = 1./(size[0])
dh = 1./(size[1])
x = (box[0] + box[1])/2.0 - 1
y = (box[2] + box[3])/2.0 - 1
w = box[1] - box[0]
h = box[3] - box[2]
x = x*dw
w = w*dw
y = y*dh
h = h*dh
return (x,y,w,h)
def convert_annotation(year, image_id):
in_file = open('VOCdevkit/VOC%s/Annotations/%s.xml'%(year, image_id))
out_file = open('VOCdevkit/VOC%s/labels/%s.txt'%(year, image_id), 'w')
tree=ET.parse(in_file)
root = tree.getroot()
size = root.find('size')
w = int(size.find('width').text)
h = int(size.find('height').text)
for obj in root.iter('object'):
difficult = obj.find('difficult').text
cls = obj.find('name').text
if cls not in classes or int(difficult)==1:
continue
cls_id = classes.index(cls)
xmlbox = obj.find('bndbox')
b = (float(xmlbox.find('xmin').text), float(xmlbox.find('xmax').text), float(xmlbox.find('ymin').text), float(xmlbox.find('ymax').text))
bb = convert((w,h), b)
out_file.write(str(cls_id) + " " + " ".join([str(a) for a in bb]) + '\n')
wd = getcwd()
for year, image_set in sets:
if not os.path.exists('VOCdevkit/VOC%s/labels/'%(year)):
os.makedirs('VOCdevkit/VOC%s/labels/'%(year))
image_ids = open('VOCdevkit/VOC%s/ImageSets/Main/%s.txt'%(year, image_set)).read().strip().split()
list_file = open('%s_%s.txt'%(year, image_set), 'w')
for image_id in image_ids:
list_file.write('%s/VOCdevkit/VOC%s/JPEGImages/%s.jpg\n'%(wd, year, image_id))
convert_annotation(year, image_id)
list_file.close()
\end{lstlisting}

\chapter{Trešais pielikums}
\label{appendix:pielikums3}
\begin{lstlisting}[basicstyle=\tiny]
#!/usr/bin/env python 
#encoding=utf8
'''
Detection with SSD
In this example, we will load a SSD model and use it to detect objects.
'''

import os
import sys
import argparse
import numpy as np
import cv2
import math
from PIL import Image, ImageDraw, ImageFont
# Make sure that caffe is on the python path:
caffe_root = './'
os.chdir(caffe_root)
sys.path.insert(0, os.path.join(caffe_root, 'python'))
import caffe

(major_ver, minor_ver, subminor_ver) = (cv2.__version__).split('.')
from google.protobuf import text_format
from caffe.proto import caffe_pb2


def get_labelname(labelmap, labels):
num_labels = len(labelmap.item)
labelnames = []
if type(labels) is not list:
labels = [labels]
for label in labels:
found = False
for i in xrange(0, num_labels):
if label == labelmap.item[i].label:
found = True
labelnames.append(labelmap.item[i].display_name)
break
assert found == True
return labelnames

class CaffeDetection:
def __init__(self, gpu_id, model_def, model_weights, image_resize, labelmap_file):
caffe.set_device(gpu_id)
caffe.set_mode_gpu()

self.image_resize = image_resize
# Load the net in the test phase for inference, and configure input preprocessing.
self.net = caffe.Net(model_def,      # defines the structure of the model
model_weights,  # contains the trained weights
caffe.TEST)     # use test mode (e.g., don't perform dropout)
# input preprocessing: 'data' is the name of the input blob == net.inputs[0]
self.transformer = caffe.io.Transformer({'data': self.net.blobs['data'].data.shape})
self.transformer.set_transpose('data', (2, 0, 1))
self.transformer.set_mean('data', np.array([104, 117, 123])) # mean pixel
# the reference model operates on images in [0,255] range instead of [0,1]
self.transformer.set_raw_scale('data', 255)
# the reference model has channels in BGR order instead of RGB
self.transformer.set_channel_swap('data', (2, 1, 0))

# load PASCAL VOC labels
file = open(labelmap_file, 'r')
self.labelmap = caffe_pb2.LabelMap()
text_format.Merge(str(file.read()), self.labelmap)

def detect(self, image_file, conf_thresh=0.24, topn=5):
'''
SSD detection
'''
# set net to batch size of 1
image_resize = 512
self.net.blobs['data'].reshape(1, 3, self.image_resize, self.image_resize)
#image = caffe.io.load_image(image_file) 
image = image_file
#Run the net and examine the top_k results
transformed_image = self.transformer.preprocess('data', image)
self.net.blobs['data'].data[...] = transformed_image

# Forward pass.
detections = self.net.forward()['detection_out']

# Parse the outputs.
det_label = detections[0,0,:,1]
det_conf = detections[0,0,:,2]
det_xmin = detections[0,0,:,3]
det_ymin = detections[0,0,:,4]
det_xmax = detections[0,0,:,5]
det_ymax = detections[0,0,:,6]

# Get detections with confidence higher than 0.6.
top_indices = [i for i, conf in enumerate(det_conf) if conf >= conf_thresh]

top_conf = det_conf[top_indices]
top_label_indices = det_label[top_indices].tolist()
top_labels = get_labelname(self.labelmap, top_label_indices)
top_xmin = det_xmin[top_indices]
top_ymin = det_ymin[top_indices]
top_xmax = det_xmax[top_indices]
top_ymax = det_ymax[top_indices]

result = []
for i in xrange(min(topn, top_conf.shape[0])):
xmin = top_xmin[i] # xmin = int(round(top_xmin[i] * image.shape[1]))
ymin = top_ymin[i] # ymin = int(round(top_ymin[i] * image.shape[0]))
xmax = top_xmax[i] # xmax = int(round(top_xmax[i] * image.shape[1]))
ymax = top_ymax[i] # ymax = int(round(top_ymax[i] * image.shape[0]))
score = top_conf[i]
label = int(top_label_indices[i])
label_name = top_labels[i]
result.append([xmin, ymin, xmax, ymax, label, score, label_name])
return result

def main(args):
'''main '''
'''defining detection and video'''
videoFile = cv2.VideoCapture(args.video)
detection = CaffeDetection(args.gpu_id,args.model_def,args.model_weights,args.image_resize,args.labelmap_file)
init_once = False
vlength = int(videoFile.get(cv2.CAP_PROP_FRAME_COUNT))    
'''Going through video'''
trackerlist = list()
boxCenters = list()
buul,firstFrame = videoFile.read()
firstFrame = cv2.resize(firstFrame, (0,0), fx=01, fy=01)
roi = cv2.selectROI(firstFrame,False)
line = cv2.selectROI(firstFrame,False)
countLine = ((line[0],line[1]),(line[0] + line[2],line[1]+line[3]))
cv2.destroyAllWindows() 
result = list()
finishcenters =list()
tracker = cv2.MultiTracker_create()  
framez = 0 
count = 0;
while(videoFile.isOpened()):  
framez = framez +1
ret, frame = videoFile.read()
if framez >= 0:
frame = cv2.resize(frame, (0,0), fx=1, fy=1)        
im = Image.fromarray(frame)      
result = detection.detect(frame)        
width, height = im.size
font = cv2.FONT_HERSHEY_SIMPLEX  
for item in result:
newTracker = False              
xmin = int(round(item[0] * width))
ymin = int(round(item[1] * height))
xmax = int(round(item[2] * width))
ymax = int(round(item[3] * height))
bbox = (int(xmin), int(ymin), int(xmax-xmin), int(ymax-ymin))
detectCenter = (int((xmin + xmax)*0.5),int((ymin+ymax)*0.5))   
if (roi[1] < detectCenter[1] < roi[1]+roi[3] and roi[0] < detectCenter[0] < roi[0]+roi[2]):
cv2.rectangle(frame,(xmin, ymin), (xmax, ymax),(255, 0, 0),3)            
cv2.putText(frame,item[-1] + str(item[-2]),(xmin,ymin), font, 0.5,(255,255,255),2,cv2.LINE_AA)                   
if not boxCenters:
tracker.add(cv2.TrackerMIL_create(), frame, bbox)
else:
newTracker = False
distanceList = list()
for cent in boxCenters:
distanceList.append(math.sqrt( ((cent[0]-detectCenter[0])**2)+((cent[1]-detectCenter[1])**2)))
if min(distanceList)>100:
newTracker = True
if newTracker:
tracker.add(cv2.TrackerMIL_create(), frame, bbox)
ok, boxes = tracker.update(frame)
tempCenters = boxCenters
boxCenters = list()
for idx,newbox in enumerate(boxes):            
p1 = (int(newbox[0]), int(newbox[1]))
p2 = (int(newbox[0] + newbox[2]), int(newbox[1] + newbox[3]))
center = ((int((newbox[0]+int(newbox[0] + newbox[2]))*0.5)),(int((newbox[1]+int(newbox[1] + newbox[3]))*0.5)))
finishcenters.append(center)
if (roi[1] < center[1] < roi[1]+roi[3] and roi[0] < center[0] < roi[0]+roi[2]) and ok:
boxCenters.append(center)
cv2.rectangle(frame, p1, p2, (0,255,0))
if (countLine[0] is not None and idx<len(tempCenters) and idx<len(boxCenters)):
if(tempCenters[idx][1]< countLine[0][1] and boxCenters[idx][1] >= countLine[0][1]) or (tempCenters[idx][1]<countLine[1][1] and boxCenters[idx][1] >= countLine[1][1]):
count = count+1
cv2.putText(frame, "Count:" + str(len(boxCenters)), (100,20), cv2.FONT_HERSHEY_SIMPLEX, 0.75, (50,170,50),2); 
cv2.line(frame,countLine[0],countLine[1],(255,0,0),1)

frame = cv2.resize(frame, (0,0), fx=0.5, fy=0.5)
cv2.imshow('frame',frame)
if cv2.waitKey(1) & 0xFF == ord('q'):

for item in finishcenters:
cv2.circle(frame, (item[0],item[1]), 3  , (0,255,0), thickness=5, lineType=1, shift=0)
#frame[item[1]][item[0]] = [255,255,255]
cv2.rectangle(frame, (roi[0], roi[1]), (roi[0]+roi[2], roi[1]+roi[3]), (255,0,0), 2)
cv2.imshow('frame',frame)
cv2.waitKey(0)
videoFile.release()
out.release()
#cv2.destroyAllWindows()

def parse_args():
'''parse args'''
parser = argparse.ArgumentParser()
parser.add_argument('--gpu_id', type=int, default=0, help='gpu id')
parser.add_argument('--labelmap_file',
default='data/VOC0712/labelmap_voc.prototxt')
parser.add_argument('--model_def',
default='models/VGGNet/VOC0712/SSD_512x512/deploy.prototxt')
parser.add_argument('--image_resize', default=512, type=int)
parser.add_argument('--model_weights',
default='models/VGGNet/VOC0712/SSD_512x512/VGG_VOC0712_SSD_512x512_iter_124340.caffemodel')
parser.add_argument('--video', default='/home/edgars/Desktop/MD/caffe/examples/videos/horiz1.mp4')
return parser.parse_args()

if __name__ == '__main__':
main(parse_args())

\end{lstlisting}
\chapter{Ceturtais pielikums}
\label{appendix:pielikums4}
\begin{lstlisting}[basicstyle=\tiny]
#=====================darknet.py============================
from ctypes import *
import math
import random

def sample(probs):
s = sum(probs)
probs = [a/s for a in probs]
r = random.uniform(0, 1)
for i in range(len(probs)):
r = r - probs[i]
if r <= 0:
return i
return len(probs)-1

def c_array(ctype, values):
return (ctype * len(values))(*values)

class BOX(Structure):
_fields_ = [("x", c_float),
("y", c_float),
("w", c_float),
("h", c_float)]

class DETECTION(Structure):
_fields_ = [("bbox", BOX),
("classes", c_int),
("prob", POINTER(c_float)),
("mask", POINTER(c_float)),
("objectness", c_float),
("sort_class", c_int)]


class IMAGE(Structure):
_fields_ = [("w", c_int),
("h", c_int),
("c", c_int),
("data", POINTER(c_float))]

class METADATA(Structure):
_fields_ = [("classes", c_int),
("names", POINTER(c_char_p))]



#lib = CDLL("/home/pjreddie/documents/darknet/libdarknet.so", RTLD_GLOBAL)
lib = CDLL("/home/edgars/Desktop/yolo/____darknet/libdarknet.so", RTLD_GLOBAL)
lib.network_width.argtypes = [c_void_p]
lib.network_width.restype = c_int
lib.network_height.argtypes = [c_void_p]
lib.network_height.restype = c_int

predict = lib.network_predict
predict.argtypes = [c_void_p, POINTER(c_float)]
predict.restype = POINTER(c_float)

set_gpu = lib.cuda_set_device
set_gpu.argtypes = [c_int]

make_image = lib.make_image
make_image.argtypes = [c_int, c_int, c_int]
make_image.restype = IMAGE

get_network_boxes = lib.get_network_boxes
get_network_boxes.argtypes = [c_void_p, c_int, c_int, c_float, c_float, POINTER(c_int), c_int, POINTER(c_int)]
get_network_boxes.restype = POINTER(DETECTION)

make_network_boxes = lib.make_network_boxes
make_network_boxes.argtypes = [c_void_p]
make_network_boxes.restype = POINTER(DETECTION)

free_detections = lib.free_detections
free_detections.argtypes = [POINTER(DETECTION), c_int]

free_ptrs = lib.free_ptrs
free_ptrs.argtypes = [POINTER(c_void_p), c_int]

network_predict = lib.network_predict
network_predict.argtypes = [c_void_p, POINTER(c_float)]

reset_rnn = lib.reset_rnn
reset_rnn.argtypes = [c_void_p]

load_net = lib.load_network
load_net.argtypes = [c_char_p, c_char_p, c_int]
load_net.restype = c_void_p

do_nms_obj = lib.do_nms_obj
do_nms_obj.argtypes = [POINTER(DETECTION), c_int, c_int, c_float]

do_nms_sort = lib.do_nms_sort
do_nms_sort.argtypes = [POINTER(DETECTION), c_int, c_int, c_float]

free_image = lib.free_image
free_image.argtypes = [IMAGE]

letterbox_image = lib.letterbox_image
letterbox_image.argtypes = [IMAGE, c_int, c_int]
letterbox_image.restype = IMAGE

load_meta = lib.get_metadata
lib.get_metadata.argtypes = [c_char_p]
lib.get_metadata.restype = METADATA

load_image = lib.load_image_color
load_image.argtypes = [c_char_p, c_int, c_int]
load_image.restype = IMAGE

rgbgr_image = lib.rgbgr_image
rgbgr_image.argtypes = [IMAGE]

predict_image = lib.network_predict_image
predict_image.argtypes = [c_void_p, IMAGE]
predict_image.restype = POINTER(c_float)

def array_to_image(arr):
print arr
arr = arr.transpose(2,0,1)
c = arr.shape[0]
h = arr.shape[1]
w = arr.shape[2]
arr = (arr/255.0).flatten()
data = c_array(c_float, arr)
im = IMAGE(w,h,c,data)
return im
def classify(net, meta, im):
out = predict_image(net, image)
res = []
for i in range(meta.classes):
res.append((meta.names[i], out[i]))
res = sorted(res, key=lambda x: -x[1])
return res

def detect(net, meta, image, thresh=.3, hier_thresh=.5, nms=.45):


im = load_image(image, 0, 0)
num = c_int(0)
pnum = pointer(num)
predict_image(net, im)
dets = get_network_boxes(net, im.w, im.h, thresh, hier_thresh, None, 0, pnum)
num = pnum[0]
if (nms): do_nms_obj(dets, num, meta.classes, nms);

res = []
for j in range(num):
for i in range(meta.classes):
if dets[j].prob[i] > 0:
b = dets[j].bbox
res.append((meta.names[i], dets[j].prob[i], (b.x, b.y, b.w, b.h)))
res = sorted(res, key=lambda x: -x[1])
free_image(im)
free_detections(dets, num)
return res
#=======================================
#==========detector.py==================
from ctypes import *
import math
import random
import os
import sys
import argparse
import numpy as np
import cv2
from PIL import Image, ImageDraw, ImageFont
sys.path.append(os.path.join(os.getcwd(),'python/'))
import natsort
import darknet as dn
import pdb
def unique(list1):
	# intilize a null list
	unique_list = []
	# traverse for all elements
	for x in list1:
	# check if exists in unique_list or not
	if x not in unique_list:
	unique_list.append(x)
	return unique_list
	
net = dn.load_net("/home/edgars/Desktop/yolo/darknet/build/darknet/x64/cfg/yolo-obj.cfg", "/home/edgars/Desktop/yolo/darknet/build/darknet/x64/backup/yolo-obj_8500.weights", 0)
meta = dn.load_meta("/home/edgars/Desktop/yolo/darknet/build/darknet/x64/data/obj.data")
im = "/home/edgars/Desktop/yolo/____darknet/data/crowd2.jpg"
#cap = cv2.VideoCapture("/home/edgars/Desktop/MD/caffe/examples/videos/ILSVRC2015_train_00755001.mp4")
directory = '/home/edgars/Downloads/train/tempvid'
allfiles = os.listdir(directory)
allfiles = natsort.natsorted(allfiles)
firstimage = '/home/edgars/Downloads/train/tempvid/'+allfiles[0]
print firstimage
firstimage2 = cv2.imread(firstimage) 
firstFrame = firstimage2#cv2.resize(firstimage2, (0,0), fx=1, fy=1)
roi = cv2.selectROI(firstFrame,False)
line = cv2.selectROI(firstFrame,False)
countLine = ((line[0],line[1]),(line[0] + line[2],line[1]+line[3]))
result = list()
tracker = cv2.MultiTracker_create()    
trackerlist = list()
boxCenters = list()
finishcenters =list()
cv2.destroyAllWindows() 
framez = 0
for filename in allfiles:
	framez = framez +1
	print framez
	if framez > 0:
		if filename.endswith(".png") or filename.endswith(".jpg") or filename.endswith(".jpeg"): 
			image = '/home/edgars/Downloads/train/tempvid/'+filename
			arr = cv2.imread(image)  
			frame = cv2.resize(arr, (0,0), fx=1, fy=1)        
			im = Image.fromarray(frame)      
			result = dn.detect(net,meta,image)
			width, height = im.size
			font = cv2.FONT_HERSHEY_SIMPLEX  
			for item in result:
				newTracker = False
				xmin = int(round(item[2][0]))
				ymin = int(round(item[2][1]))
				xmax = int(round(item[2][0])+round(item[2][2]))
				ymax = int(round(item[2][1])+round(item[2][3]))
				bbox = (int(xmin), int(ymin), int(xmax-xmin), int(ymax-ymin))
				detectCenter = (int((xmin + xmax)*0.5),int((ymin+ymax)*0.5))
				finishcenters.append(detectCenter)
				if (roi[1] < detectCenter[1] < roi[1]+roi[3] and roi[0] < detectCenter[0] < roi[0]+roi[2]):
					cv2.rectangle(frame,(xmin, ymin), (xmax, ymax),(255, 0, 0),3)
					cv2.putText(frame,item[-1] + str(item[-2]),(xmin,ymin), font, 0.5,(255,255,255),2,cv2.LINE_AA)
					if not boxCenters:
						tracker.add(cv2.TrackerMIL_create(), frame, bbox)
					else:
						newTracker = False
						distanceList = list()
						if boxCenters is not None:
							boxCenters = unique(boxCenters)
						for cent in boxCenters:
							distanceList.append(math.sqrt( ((cent[0]-detectCenter[0])**2)+((cent[1]-detectCenter[1])**2)))
						if min(distanceList)>150:
							newTracker = True
				if newTracker:
					tracker.add(cv2.TrackerMIL_create(), frame, bbox)
					ok, boxes = tracker.update(frame)
					tempCenters = boxCenters
					boxCenters = list()
				for idx,newbox in enumerate(boxes):
					p1 = (int(newbox[0]), int(newbox[1]))
					p2 = (int(newbox[0] + newbox[2]), int(newbox[1] + newbox[3]))
					center = ((int((newbox[0]+int(newbox[0] + newbox[2]))*0.5)),(int((newbox[1]+int(newbox[1] + newbox[3]))*0.5)))
				if (roi[1] < center[1] < roi[1]+roi[3] and roi[0] < center[0] < roi[0]+roi[2]) and ok:
					boxCenters.append(center)
					cv2.rectangle(frame, p1, p2, (0,255,0))
				if ok is not True:
					tracker = cv2.MultiTracker_create() 
				if (countLine[0] is not None and idx<len(tempCenters) and idx<len(boxCenters)):
					if(tempCenters[idx][1]< countLine[0][1] and boxCenters[idx][1] >= countLine[0][1]) or (tempCenters[idx][1]<countLine[1][1] and boxCenters[idx][1] >= countLine[1][1]):
					count = count+1

cv2.putText(frame, "Count:" + str(len(boxCenters)), (100,20), cv2.FONT_HERSHEY_SIMPLEX, 0.75, (50,170,50),2); 
cv2.rectangle(frame, (roi[0], roi[1]), (roi[0]+roi[2], roi[1]+roi[3]), (255,0,0), 2)
cv2.line(frame,countLine[0],countLine[1],(255,0,0),1)
for item in finishcenters:
	cv2.circle(frame, (item[0],item[1]), 3  , (0,255,0), thickness=5, lineType=1, shift=0)
	cv2.imshow('frame',frame)
	cv2.waitKey(0)
#=======================================
\end{lstlisting}
%% Vēl jāpievieno atzīmes lapa
\pagestyle{empty}

\addtocontents{toc}{\protect\enlargethispage{20\baselineskip}}
\chapter*{Galvojums}
\addcontentsline{toc}{chapter}{Galvojums}
Ar šo es, \defAutors, galvoju, ka maģistra darbs ir izpildīts patstāvīgi un bez citu palīdzības. No svešiem pirmavotiem ņemtie dati un definējumi ir uzrādīti darbā. Šis darbs tādā vai citādā veidā nav nekad iesniegts nevienai citai pārbaudījumu komisijai un nav nekur publicēts.

Esmu informēts (-a), ka mans maģistra darbs tiks ievietots un apstrādāts Vienotajā datorizētajā plaģiāta kontroles sistēmā plaģiāta kontroles nolūkos.

\vspace{2cm}
\defGads.gada \rule{1cm}{0.2pt}.\rule{3cm}{0.2pt}\hspace{4cm}\rule{4cm}{0.2pt}
\vspace{1cm}

Es, \defAutors, atļauju Ventspils Augstskolai savu maģistra darbu bez atlīdzības ievietot un uzglabāt Latvijas Nacionālās bibliotēkas pārvaldītā datortīklā Academia\\ (www.academia.lndb.lv), kurā tie ir pieejami gan bibliotēkas lietotājiem, gan globālajā tīmeklī tādā veidā, ka ikviens tiem var piekļūt individuāli izraudzītā laikā, individuāli izraudzītā vietā.


\begin{flushright}
	Piekrītu \rule{3cm}{0.2pt}
	\vspace{1.5cm}
	
	Nepiekrītu \rule{3cm}{0.2pt}
	
\end{flushright}

\vspace{1cm}
\defGads.gada \rule{1cm}{0.2pt}.\rule{3cm}{0.2pt}

\pagestyle{empty}

\newpage
\begin{center}
 Maģistra darbs aizstāvēts Valsts pārbaudījumu komisijas sēdē\\
 \vspace{1em}
\end{center}
\defGads.gada \rule{1cm}{0.2pt} . \rule{3cm}{0.2pt}\\\\
un novērtēts ar atzīmi \rule{4cm}{0.2pt} \\\\\\
Protokols Nr. \rule{1cm}{0.2pt}\\\\\\
Valsts pārbaudījumu komisijas \\\\
priekšsēdētājs \rule{7cm}{0.2pt}.\\
\hspace*{5cm}\textit{\raisebox{1em}{paraksts}}


\end{document}
